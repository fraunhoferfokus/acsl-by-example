
\section{The \removecopyiii algorithm (final contract)}
\label{sec:removecopyiii}

In this section we extend the contracts of \specref{removecopy}
and \specref{removecopyii} by introducing a 
logic function, which describes the relationship between the elements of
input range \inl{a[0..n-1]} and the output range \inl{b[0..\\result-1]}.
Note that we have shown in the previous section that
\inl{\\result} equals \inl{CountNotEqual(a, n, v)}.

\subsection{A closer look on the properties of \removecopy}
\label{sec:formal-view-remove}

Figure~\ref{fig:removecopy-trip} shows a modified version of the
Figure~\ref{fig:removecopy}. We left out the indices of values that were not copied into the
target array. Furthermore we have added a dashed arrow which points to the
index that corresponds to the \emph{one past the end} location of the input and
output range.

\begin{figure}[hbt]
\begin{center}
\includegraphics[width=0.70\textwidth]{Figures/remove_copy_partition.pdf}
\end{center}
\caption{\label{fig:removecopy-trip} Partitioning the input of \removecopy}
\end{figure}
\FloatBarrier

These arrows between the indices of the array~\inl{b} and array~\inl{a} define
the following sequence~$p$ of seven indices. The index of the \emph{one past the end} is underlined. 
$p = (1, 2, 5, 7, 8, 10, \underline{11})$


%\clearpage 

More generally, we refer to the sequence~$p$ as \emph{partitioning sequence} of \removecopy for the array \inl{a[0..n-1]}.
For the \textbf{length of a partitioning sequence} $m$ we get the following \textbf{strictly monotone increasing} sequence:
\begin{align}
  \label{eq:remove-monotone}
  0 &\leq  p_0 < ... < p_{m} = n \\
  \intertext{and the open index intervals}
  %\label{eq:remove-interval}
  \nonumber
  (p_i,&p_{i+1}) && \forall i: 0 \leq i < m\\
  \intertext{mark \textbf{consecutive ranges} of the value \inl{v}
  in the source array, that is,}
  \label{eq:between}
  a[k] &= v &&\forall k: p_i < k < p_{i+1}\\
  \intertext{Additionally, the half open interval}
  \nonumber
  [0,&p_{0})\\ 
  \intertext{also marks another \textbf{consecutive range} of the value \inl{v} in the
  source array:}
  \label{eq:beginning}
  a[k] &= v &&\forall k: 0 \leq k < p_{0}\\
  \intertext{Another observation is that} 
  \label{eq:a_nv}
  a[p_i] &\neq v &&\forall i: 0 \leq i < m\\
  \intertext{holds. Finally, we have}
  \label{eq:b_eqa}
  a[p_i] &= b[i] &&\forall i: 0 \leq i < m\\
  \intertext{which, together with the inequality~\eqref{eq:a_nv} 
  states, that the target does not contain the value~\inl{v}}
  \nonumber
  %\label{eq:final}
  b[i] &\neq v &&\forall i:0 \leq i < m 
\end{align} 

\subsection{More lemmas on \CountNotEqual}

Our formalization the properties of \S\ref{sec:formal-view-remove}
relies on the logic function \logicref{CountNotEqual}.
We also rely on the logic function \logicref{FindNotEqual} and
the lemmas of \logicref{CountFindNotEqual} in the following listing that provide more
facts about \CountNotEqual and \FindNotEqual.

\input{Listings/CountFindNotEqual.acsl.tex}

\clearpage

\subsection{Formalizing the properties of the partitions}

The function~\RemovePartition, whose axiomatic definition is given in 
Listings~\ref{logic:RemovePartition-1} and~\ref{logic:RemovePartition-2}
defines the partitioning sequence~$p$ from \S\ref{sec:formal-view-remove}.


\begin{logic}[hbt]
\begin{minipage}{\textwidth}
\lstinputlisting[linerange={1-53}, style=acsl-block, frame=single]{Source/RemovePartition.acsl}
\end{minipage}
\caption{\Label{logic:RemovePartition-1} The logic function \RemovePartition (1)}
\input{Listings/RemovePartition.acsl.labels.tex}
\input{Listings/RemovePartition.acsl.index.tex}
\end{logic}

\FloatBarrier

Before we begin to relate the various lemmas
to the formulas from \S\ref{sec:formal-view-remove} we want to remind
the reader that logic functions (and predicates) must be total that is they must
be defined for all possible argument values.

\begin{logic}[hbt]
\begin{minipage}{\textwidth}
\lstinputlisting[linerange={54-103}, style=acsl-block, frame=single]{Source/RemovePartition.acsl}
\end{minipage}
\caption{\Label{logic:RemovePartition-2}The logic function \RemovePartition (2)}
\end{logic}

\FloatBarrier

The lemmas for \RemovePartition are related to the properties of 
\S\ref{sec:formal-view-remove} in the following way.

\begin{itemize}
  \item Property \eqref{eq:remove-monotone} is expressed by the lemmas
  \RemovePartitionEmpty, \RemovePartitionLeft
  \RemovePartitionRight, and \RemovePartitionStrictlyWeakIncreasing

  \item Properties~\eqref{eq:between} and~\eqref{eq:beginning}
  are described by lemmas \RemovePartitionSegment.

  \item Property~\eqref{eq:a_nv} is expressed by lemma \RemovePartitionNotEqual.

  \item Property~\eqref{eq:b_eqa} is formulated using the predicate \logicref{Remove}.
\end{itemize}

\clearpage

We would like to point out lemma \RemovePartitionCore which subsumes
the statements of the subsequent lemmas \RemovePartitionUpper,
\RemovePartitionNotEqual,\\
and \RemovePartitionCount.
While these three lemmas add nothing new 
we have kept them because they correspond directly to individual properties
of \S\ref{sec:formal-view-remove}.
The question may arise why there is the lemma \RemovePartitionCore in the first place.
The answer is that we found the individual properties so intertwined
that we were not able to verify them separately but only their joint embodiment.


\subsection{The predicate \Remove}

The predicate \logicref{Remove} primarily serves 
in order to improve the readability of our specification \specref{removecopyiii}.
As mentioned before this predicate encapsulates the Property~\eqref{eq:b_eqa}
from \S\ref{sec:formal-view-remove}.
Note that \logicref{Remove} also contains an overloaded version of
\Remove which will be used for the specification of the \emph{in-place}
variant \specref{remove} of \removecopy.

\input{Listings/Remove.acsl.tex}

\clearpage

\subsection{Formal specification of \removecopyiii}

The following listing shows the formal specification of \specref{removecopy}.
The additional postcondition \inl{remove} makes use of the predicate \logicref{Remove}
which we have just described.
Furthermore, we have again the postcondition~\inl{unchanged} which states that
the source array~\inl{a[0..n-1]} does not change.

\input{Listings/remove_copy3.h.tex}

\subsection{Implementation of \removecopyiii}
\label{sec:removecopyiii:impl}

We discuss now some aspects of the implementation of \implref{removecopyiii}.
We introduce the loop invariant~\inl{mapping}.
This invariant states that the variable~\inl{i} will
always be smaller or equal to the result of \inl{RemovePartition(a, n, v, k)}.
We also add the assertion~\inl{mapping} to our implementation as stepping stone
for the provers to verify the correctness of this loop invariant.

Somewhat surprisingly, in order to reduce excessive verification times we had to add
an else-branch to our implementation that besides the assertion \inl{unchanged}
is empty.

Regarding the assertion \inl{update}, one might wonder why we do not simply write
\inl{\\at(a[i], Pre)}. 
However, this expression would be wrong because the index~\inl{i}
would then be interpreted as
\inl{\\at(i,Pre)} which doesn't makes sense for a local variable.
\wpframac consequently rejects this expression with the following error message.

\begin{small}
\begin{verbatim}
       Warning: unbound logic variable i. Ignoring code annotation
\end{verbatim}
\end{small}

\clearpage

We could explicitly refer to the current value of~\inl{i} by using
the subexpression \inl{\\at(i,Here)} inside the assertion~\inl{update}.
We felt, however, tow introduce the predicate \logicref{At}
to simplify the comparison of array elements in programme states 
where the particular index variable isn't visible.

\input{Listings/At.acsl.tex}

The second argument \At is interpreted at the programme point here it appears,
that is, \inl{Here}.
Using this auxiliary logic function the assertion \inl{update}
is arguably more readable.

\input{Listings/remove_copy3.c.tex}

\clearpage

