
\section{The \reversecopy algorithm}
\Label{sec:reversecopy}

The \reversecopy
algorithm of the \cxx Standard Library \cite[\S 28.6.10]{cxx-17-draft} inverts the order of elements
in a sequence.
\reversecopy does not change the input sequence, and
copies its result to the output sequence.
For our purposes we have modified the generic implementation
to that of a range of type \valuetype.
The signature now reads:

\begin{lstlisting}[style=acsl-block]

  void reverse_copy(const value_type* a, size_type n, value_type* b);
\end{lstlisting}

%\subsection{The predicate \Reverse}

Informally, \reversecopy copies the elements from the array \inl{a} into
array \inl{b} such that the copy is a reverse of the original array. 
In order to concisely formalize these conditions we define in the following
listing the predicate \logicref{Reverse} (see also Figure~\ref{fig:Reverse}).

\input{Listings/Reverse.acsl.tex}

We also define several overloaded variants of \Reverse that 
provide default values for some of the parameters.
These overloaded versions enable us to write later more concise \acsl annotations.

\begin{figure}[hbt]
\centering
\includegraphics[width=0.60\textwidth]{Figures/reverse_logic.pdf}
\caption{\Label{fig:Reverse} Sketch of predicate~\Reverse}
\end{figure}

\FloatBarrier

\subsection{Formal specification of \reversecopy}

The specification of \specref{reversecopy} is shown in the following listing
We use the second version of predicate \logicref{Reverse}
in order to formulate the postcondition of \reversecopy.

\input{Listings/reverse_copy.h.tex}

\subsection{Implementation of \reversecopy}

The implementation of \implref{reversecopy} is shown in the following listing.
%
For the postcondition to be true, we must ensure that for
every element \inl{i}, the comparison \inl{b[i] == a[n-1-i]} holds.
This is formalized by the loop invariant~\inl{reverse} where we employ
the first version of \logicref{Reverse}.

\input{Listings/reverse_copy.c.tex}

\clearpage

