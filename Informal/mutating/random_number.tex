
\section{Verifying a random number generator}
\Label{sec:randomnumber}

We describe in this section \specref{randomnumber} which implements
a simple random-number generator.
As in the case of \specref{shuffle} itself, we do not formulate precise
properties of randomness and only require its result to
be in the specified range \inl{[0..n-1]}.
Again, the \inl{assigns} clause to the array \inl{state} models the
dependency on an additional state.

Note that in the following listing, we also provide the rather simple
specification of the function \randominit that is called to initialize the 
state of the random generator.

\input{Listings/random_number.h.tex}

The implementations of \randomnumber and \randominit are shown in the following listing.
Internally, we rely on a custom implementation of the POSIX.1
random number generator \inl{lrand48()}\footnote{
  See \url{http://pubs.opengroup.org/onlinepubs/9699919799/functions/lrand48.html}
}
This random number generator is a linear congruence
generator with a 48~bit state and the iteration procedure
\begin{equation}
\label{eq:random}
x_{n+1}=ax_n+c\bmod 2^{48}
\end{equation}
%
where $a=25214903917$ and $c=11$ are relatively prime integers.

As a part of the iteration procedure in Equation~\eqref{eq:random}
an unsigned overflow may occur.
This does not affect the result as we are only interested in its lowest 48 bits.
However, as one of the options we use, \inl{-warn-unsigned-overflow},
causes \wpframac  assert the absence of unsigned overflow this algorithm does not verify under
the same options used for the other algorithms.
%
As an exception, we have therefore decided to disable \inl{-warn-unsigned-overflow}
for this function as the unsigned overflow is both benign and
well-defined (cf.\ \cite[\S 6.2.5, 9]{isoc}).

\input{Listings/random_number.c.tex}

Note that we use the custom acsl lemma \logicref{RandomNumberModulo}
from the following listing to support the verification of some assertions.

\input{Listings/C_Bit.acsl.tex}

\clearpage

