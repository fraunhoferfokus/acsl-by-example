
\section{The predicate \Unchanged}
\Label{sec:unchanged}

Many of the algorithms in this section iterate sequentially over one or several sequences.
For the verification of such algorithms it is often important to express that a section
of an array, or the complete array, have remained \emph{unchanged}.
As this cannot always be expressed by an \inl{assigns} clause,
we introduce in the following listing the overloaded predicate
\logicref{Unchanged}.
The expression \inl{Unchanged\{K,L\}(a,m,n)} is true if the range
\inl{a[m..n-1]} in state~\inl{K} is element-wise equal to that range in state~\inl{L}.

\input{Listings/Unchanged.acsl.tex}

In some situations we use the predicate \ArrayUpdate,
which relies on the predicate \Unchanged and the the logic function \logicref{At},
to concisely describe which parts of an array have changed or remained unchanged
when updating an individual array element.


\input{Listings/ArrayUpdate.acsl.tex}

\clearpage

In the following listing we show a few lemmas for \logicref{Unchanged}
that we need for the verification of various algorithms.

\input{Listings/UnchangedLemmas.acsl.tex}

\begin{itemize}
\item
Lemma~\logicref{UnchangedShrink} states that if the range~\inl{a[m..n-1]} does
not change when going from state~\inl{K} to state~\inl{L}, then~\inl{a[p..q-1]}
does not change either, provided the latter is a subrange of the
former, i.e.\ provided $0 \leq m \leq p \leq q \leq n$ holds.

\item
Lemma~\logicref{UnchangedExtend} expresses the simple fact that ``unchangedness'' is an inductive property.

\item
Lemma~\logicref{UnchangedShift} states how \Unchanged behaves under pointer additions.

\item
Lemmas~\logicref{UnchangedSymmetric} and~\logicref{UnchangedTransitive} express respectively
the symmetry and transitivity of \Unchanged with respect to program states.

\end{itemize}


\clearpage

