
\section{The \removecopy algorithm (basic contract)}
\Label{sec:removecopy}

The \removecopy algorithm of the \cxx Standard Library \cite[\S 28.6.8]{cxx-17-draft}
copies all elements of a sequence other than a given value.
Here, the general implementation has been altered to process \valuetype ranges.
The new signature reads:

\begin{lstlisting}[style=acsl-block]

  size_type
  remove_copy(const value_type* a, size_type n, value_type* b, value_type v);
\end{lstlisting}

The requirements of \removecopy are:

\begin{table}[hbt]
  \begin{center}
    \begin{tabular}{|l|p{0.6\textwidth}|}
\hline
\textbf{Requirements} & \textbf{Description}
\\\hline
\hline
\namedlabel{itm:remove-size}
{\textbf{Remove Copy Size}} &
The output range has to fit in all the elements of
the input range, except the ones that equal the value~\inl{v}
        by \removecopy.
\\\hline
        \namedlabel{itm:remove-separation}{\textbf{Remove Copy Separated}} &
        The input range and the output range do not overlap
\\\hline
        \namedlabel{itm:remove-elements}{\textbf{Remove Copy Elements}} &
        The \removecopy algorithm copies elements that
        are not equal to \inl{v}
        from range
        \inl{a[0..n-1]} to the range {\inl{b[0..\\result-1]}}.
\\\hline
        \namedlabel{itm:remove-order}{\textbf{Remove Copy Stability}} &
        The algorithm is stable, that is, the
        relative order of the elements in \inl{b} is the same as in \inl{a}.
\\\hline
        \namedlabel{itm:remove-return}{\textbf{Remove Copy Return}} &
        The return value is the length of the resulting range.                  
\\\hline
        \namedlabel{itm:remove-complexity}{\textbf{Remove Copy Complexity}} &
        The algorithm takes $n$ comparisons in every case.
\\\hline
      \end{tabular}
    \end{center}
  \caption{\label{tbl:remove_copy_props}Properties of \removecopy}
\end{table}
\FloatBarrier

Figure~\ref{fig:removecopy} shows an example of how \removecopy is supposed
to copy elements that differ from~\inl{4} from the input range to the output range.

\begin{figure}[hbt]
\centering
\includegraphics[width=0.75\textwidth]{Figures/remove_copy.pdf}
\caption{\Label{fig:removecopy}Effects of \removecopy}
\end{figure}

\FloatBarrier

\subsection{Formal specification of \removecopy}

The following listing shows our first attempt to specify \removecopy.
In postcondition~\inl{discard} we use of the predicate \logicref{NoneEqual}
to show that the value \inl{v} does not occur in the range \inl{b[0..\\result]}.

\input{Listings/remove_copy.h.tex}

One shortcoming of this specification is that the postcondition \inl{bound}
only makes very general and not very precise statements about the number of copied elements.
We will address this problem in the contract of \specref{removecopyii}.
A more serious shortcoming is, however, that we haven't specified what
the relationship between the elements of the input range \inl{a[0..n-1]}
and the output range \inl{b[0..\\result-1]} looks like.
This problem will be tackled in the contract of \specref{removecopyiii}.

%\clearpage

\subsection{Implementation of \removecopy}

An implementation of \removecopy is shown in the following listing.

\input{Listings/remove_copy.c.tex}

Here we also need to add another loop invariant~\inl{discard} which basically 
checks if \inl{v} occurs in \inl{b[0..k]} for each iteration of the loop.

%\clearpage

