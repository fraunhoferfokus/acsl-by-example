
\section{The \replacecopy algorithm}
\Label{sec:replacecopy}

The \replacecopy algorithm of the \cxx Standard Library \cite[\S 28.6.5]{cxx-17-draft} substitutes
specific elements from general sequences.
%
Here, the general implementation
has been altered to process \valuetype ranges.
The new signature reads:

\begin{lstlisting}[style=acsl-block]

  size_type replace_copy(const value_type* a, size_type n, value_type* b,
                         value_type v, value_type w);
\end{lstlisting}

The \replacecopy algorithm copies the elements from the range \inl{a[0..n]}
to range {\inl{b[0..n]}}, substituting every occurrence of \inl{v} by \inl{w}.
The return value is the length of the range.
As the length of the range is already a parameter of
the function this return value does not contain new
information.


\begin{figure}[hbt]
\centering
\includegraphics[width=0.50\textwidth]{Figures/replace.pdf}
\caption{\Label{fig:replace} Effects of \replace}
\end{figure}

Figure~\ref{fig:replace} shows the behavior of \replacecopy at hand of an example
where all occurrences of the value~3 in~\inl{a[0..n-1]} are replaced with the
value~2 in~\inl{b[0..n-1]}.


\subsection{The predicate \Replace}

We start with defining in the following listing the predicate \logicref{Replace}
that describes the intended relationship between the input array \inl{a[0..n-1]}
and the output array \inl{b[0..n-1]}.
Note the introduction of \emph{local bindings} \inl{\\let ai = ...}
and \inl{\\let bi = ...} in the definition of \Replace (see \cite[\S 2.2]{ACSLSpec}).

\input{Listings/Replace.acsl.tex}

This listing also contains a second, overloaded version of \Replace
which we will use for the specification of the related in-place
algorithm \specref{replace}.

%\clearpage

\subsection{Formal specification of \replacecopy}

Using predicate \Replace the specification of \specref{replacecopy}
is as simple as shown in the following listing.
Note that we also require that the input range \inl{a[0..n-1]} and
output range \inl{b[0..n-1]} do not overlap.

\input{Listings/replace_copy.h.tex}

\subsection{Implementation of \replacecopy}

The implementation (including loop annotations) of \implref{replacecopy}
is shown in the following listing.
Note how the structure of the loop annotations resembles
the specification of \specref{replacecopy}.

\input{Listings/replace_copy.c.tex}

\clearpage

