
\section{The \copybackward algorithm}
\Label{sec:copybackward}

The \copybackward  algorithm in the \cxx Standard Library \cite[\S 28.6.1]{cxx-17-draft} implements
another duplication algorithm for general sequences.
For our purposes we have modified
the generic implementation
to that of a range of type \valuetype.
The signature now reads:

\begin{lstlisting}[style=acsl-block]

  void copy_backward(const value_type* a, size_type n, value_type* b);
\end{lstlisting}

The main reason for the existence of \copybackward is to allow
copying when the start of the destination range \inl{a[0..n-1]} is contained
in the source range \inl{b[0..n-1]}.
In this case, \copyi can't be employed since its precondition
\inl{sep} is violated, as can be seen in the contract of \specref{copyi}.

The informal specification of \copybackward states that copying
starts at the end of the source range.
For this to work, however, the pointer \inl{b+n} must not be contained
in the source range.
Note that the order of operation (or procedure) calls cannot be
specified in \acsl.\footnote{
       The \textsf{Aoraï} specification language and the
       corresponding \framac plugin are provided to specify and verify
       temporal properties of code; however, they are beyond the scope
       of this tutorial.
}
A similar remark about order of operations tacitly applied
to earlier functions
as well, e.g.\ to \copyi, where the \cxx order was prescribed by confining
the signature to a \inl{ForwardIterator}.


Figure~\ref{fig:copybackward} gives an example where \copybackward, but \emph{not} \copyi,
can be applied.

\begin{figure}[hbt]
\centering
\includegraphics[width=0.60\textwidth]{Figures/copy_backward.pdf}
\caption{\Label{fig:copybackward} Possible overlap of \copybackward ranges}
\end{figure}

Note that in the original signature the argument~\inl{b} refers to one 
past the end of the destination range.
Here, however, it refers to its start.
The reason for this change is that in \cxx \copybackward is defined
for \emph{bidirectional iterators} which do not provide
random access operations such as adding or subtracting an index.
Since our \isoc version works on pointers we do not consider it
as necessary to use the one past the end pointer.

%\clearpage

\subsection{Formal specification of \copybackward}

The specification of \copybackward is shown in the following listing.
The \copybackward algorithm expects that the ranges \inl{a[0..n-1]} and \inl{b[0..n-1]}
are valid for reading and writing, respectively.
Precondition \inl{sep} formalizes the constraints on the overlap of
the source and destination ranges as discussed at the beginning of this section.

\input{Listings/copy_backward.h.tex}

The function \copybackward assigns the elements from the source
range \inl{a} to the destination range \inl{b}, modifying the memory of the
elements pointed to by \inl{b}.
Again, we can use the \logicref{Equal} predicate to express that the
array~\inl{a} equals~\inl{b} after \copybackward has been called.


\subsection{Implementation of \copybackward}

The following listing shows an implementation of the
\copybackward function.

\input{Listings/copy_backward.c.tex}

We have loop invariants similar to \copyi, stating the loop variable's range
(\inl{bound})
and the area that has already been copied in each cycle (\equal).

\clearpage

