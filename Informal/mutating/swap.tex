
\section{The \swap algorithm}
\Label{sec:swap}

The \swap algorithm \cite[\S 28.6.3]{cxx-17-draft}
in the \cxx Standard Library exchanges the contents of
two variables.
Similarly, the \iterswap algorithm \cite[\S 28.6.3]{cxx-17-draft}
exchanges the contents referenced by two pointers.  
Since \isoc and hence \acsl, does not support an \inl{&} type constructor (``declarator''),
we will present an algorithm that processes pointers
and refer to it as \swap.


\subsection{Formal specification of \swap}

The contract of \specref{swap} is shown in the following listing.
The preconditions state that both pointer arguments of \swap must be dereferenceable.

\input{Listings/swap.h.tex}

Upon termination of \swap the entries must be mutually exchanged.
The expression \inl{\\old(*p)} refers to the value of \inl{*p}
before \swap has be called.
By default, a postcondition refers the values after the functions has been terminated.

\subsection{Implementation of \swap}

The following listing shows the straight-forward implementation of \implref{swap}.
No interspersed \acsl annotations are needed achieve a verification by \wpframac.

\input{Listings/swap.c.tex}

\clearpage

