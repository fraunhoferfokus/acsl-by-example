
\section{The \remove algorithm}
\Label{sec:remove}

The \cxx Standard Library also
contains a function \remove\cite[28.6.8]{cxx-17-draft} performing
the same operation as \removecopy as an in-place algorithm.
Its signature is very similar to that of \removecopy,
except that there is no need for an output array.

\begin{lstlisting}[style=acsl-block]

  size_type remove(value_type* a, size_type n, value_type v);
\end{lstlisting}

Figure~\ref{fig:remove} shows how \remove is supposed
to remove all occurrences of the given value~4 from a range.

\begin{figure}[hbt]
\centering
\includegraphics[width=0.85\textwidth]{Figures/remove.pdf}
\caption{\Label{fig:remove}Effects of \remove}
\end{figure}

\FloatBarrier

\subsection{Formal specification of \remove}

The following listing shows a formal specification of the function \specref{remove}.
Our specification is very similar to the one of \specref{removecopyiii}
except that we using a version of \logicref{Remove} that takes only one pointer argument.

\input{Listings/remove.h.tex}

\clearpage

\subsection{Implementation of \remove}

In the following listing we show our implementation of \implref{remove} together with
the additional loop annotations.
Again, the annotations are very similar to those of the
implementation of \implref{removecopyiii}.

\input{Listings/remove.c.tex}

Also note the use of the predicate \logicref{At} in the loop invariant \inl{unchanged}
and the assertion \inl{update}.

\clearpage

