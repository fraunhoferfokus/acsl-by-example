
\section{The \filli algorithm}
\Label{sec:filli}

The \filli  algorithm in the \cxx
Standard Library \cite[\S 28.6.6]{cxx-17-draft} initializes general
sequences with a particular value.
The signature of our modified variant reads:

\begin{lstlisting}[style=acsl-block]

  void fill(value_type* a, size_type n, value_type v);
\end{lstlisting}

\subsection{Formal specification of \filli}

The following listing shows the formal specification of \specref{filli}.
We can express the postcondition of \filli simply by using the overloaded
predicate \logicref{AllEqual}.

\input{Listings/fill.h.tex}

The \inl{assigns}-clauses formalize that \filli modifies only the
entries of the range \inl{a[0..n-1]}.
In general, when more than one \emph{assigns clause} appears
in a function's specification,
it is permitted to modify any of the referenced memory locations.
However, if no \emph{assigns clause} appears at all,
the function is free to modify any memory location, see
\cite[\S 2.3.2]{ACSLSpec}.
To forbid a function to do any modifications outside its
scope, a clause \inl{assigns \\nothing;}
must be used, as we practised
in the example specifications in Chapter~\ref{cha:non-mutating}.

\subsection{Implementation of \filli}

The implementation of \implref{filli} comes with the loop invariant
\inl{constant} expresses that for each iteration the array is
\emph{filled} with the value of \inl{v} up to the index \inl{i} of the iteration. 
Note that we use here again the predicate \logicref{AllEqual}.

\input{Listings/fill.c.tex}

%\clearpage

