
\section{The \swapranges algorithm}
\Label{sec:swapranges}

The \swapranges algorithm
in the \cxx Standard Library \cite[\S 28.6.3]{cxx-17-draft} exchanges the
contents of two expressed ranges element-wise.
%
After translating \cxx reference types and iterators to \isoc,
our version of the original signature reads:

\begin{lstlisting}[style=acsl-block]

  void swap_ranges(value_type* a, size_type n, value_type* b);
\end{lstlisting} 
%
We do not return a value since it would equal \inl{n}, anyway.


\subsection{Formal specification of \swapranges}

The following listing shows a specification for the \specref{swapranges} algorithm.

\input{Listings/swap_ranges.h.tex}

The \swapranges algorithm works correctly only if \inl{a} and \inl{b} do not overlap.
Because of that fact we use the clause \inl{sep} to
tell \framac that \inl{a} and \inl{b} must not overlap.

With the \inl{assigns}-clause we postulate
that the \swapranges algorithm alters the elements contained
in two distinct ranges, modifying the corresponding
elements and nothing else.

The postconditions of \swapranges specify that the content
of each element in its post-state must equal the pre-state
of its counterpart.
We can use the predicate \logicref{Equal} together with the
label \inl{Old} and \inl{Here} to express the postcondition of \swapranges.
In our specification, for example, we specify that the array \inl{a} in the memory
state that corresponds to the label \inl{Here} is equal
to the array~\inl{b} at the label \inl{Old}.
Since we are specifying a postcondition \inl{Here} refers to the post-state
of \swapranges whereas \inl{Old} refers to the pre-state.

\clearpage

\subsection{Implementation of \swapranges}

The implementation of \implref{swapranges} together with the necessary
loop annotations is shown in the following listing.
Unsurprisingly, we are repeatedly calling \specref{swap}.

\input{Listings/swap_ranges.c.tex}

For the postcondition \specref{swapranges} to hold, 
our loop invariants must ensure that at each iteration all of the
corresponding elements that have already been visited are swapped.

Note that there are two additional loop invariants which claim
that all the elements that have not visited yet equal their original values.
This annotation allows us to prove the postconditions of \swapranges
despite the fact that the loop assigns is coarser than it should be.
The predicate \logicref{Unchanged} is used to express this property.

\clearpage

