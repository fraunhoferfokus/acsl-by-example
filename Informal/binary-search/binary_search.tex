\section{The \binarysearch algorithm}
\Label{sec:binarysearch}
\Label{sec:binarysearchii}


The \binarysearch algorithm is one of the four binary search
algorithms of the \cxx Standard Library \cite[\S 28.7.3.4]{cxx-17-draft}.
For our purposes we have modified
the generic implementation
to that of an array of type \valuetype.
The signature now reads:

\begin{lstlisting}[style = acsl-block]
  bool binary_search(const value_type* a, size_type n, value_type  v);
\end{lstlisting}

Again, \binarysearch requires that its input array is in increasing order.
It will return \inl{true} if there exists an index~\inl{i} 
in~\inl{a} such that \inl{a[i] == v} holds.\footnote{%
   To be more precise: The \cxx Standard Library requires that 
   \inl{(a[i] <= v)  && (v <= a[i])} holds.
   For our definition of \valuetype (see \S\ref{sec:frequentPattern}) this
   means that \inl{v} equals \inl{a[i]}.
}

\begin{figure}[hbt]
\centering
\includegraphics[width=0.60\textwidth]{Figures/binary_search.pdf}
\caption{\Label{fig:binarysearch}Some examples for \binarysearch}
\end{figure}

\FloatBarrier

In Figure~\ref{fig:binarysearch} we do not need to use arrows to visualize the
effects of \binarysearch.
The colors orange and grey of the sought-after values indicate whether the algorithm
returns true or false, respectively.

\subsection{Formal specification of \binarysearch and \binarysearchii}

The \acsl specification of \specref{binarysearch} is shown in the following listing.

\input{Listings/binary_search.h.tex}

Note that instead of the somewhat lengthy existential quantification
of \specref{binarysearch} we can use our previously introduced predicate
\logicref{SomeEqual} in order to achieve the following more concise
formal specification \specref{binarysearchii}.


\input{Listings/binary_search2.h.tex}

It is interesting to compare the specification of \specref{binarysearch}
with that of \specref{findii}.
Both algorithms allow to determine whether a value is contained in an array.
The fact that the \cxx Standard Library requires that \find has
\emph{linear} complexity whereas \binarysearch must have a
\emph{logarithmic} complexity can currently not be expressed with \acsl.


\subsection{Implementation of \binarysearch}

Our implementation \implref{binarysearchii} first calls \specref{lowerbound}.
Remember that if the latter returns an index \inl{0 <= i < n},
then we can be sure that \inl{v <= a[i]} holds.

\input{Listings/binary_search2.c.tex}

