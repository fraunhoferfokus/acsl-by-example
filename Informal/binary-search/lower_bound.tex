
\section{The \lowerbound algorithm}
\Label{sec:lowerbound}

The \lowerbound algorithm is one of the four binary search algorithms
of the \cxx Standard Library \cite[\S 28.7.3.1]{cxx-17-draft}.
For our purposes we have modified
the generic implementation
to that of an array of type \valuetype.
The signature now reads:

\begin{lstlisting}[style = acsl-block]

  size_type
  lower_bound(const value_type* a, size_type n, value_type v);
\end{lstlisting}

As with the other binary search algorithms \lowerbound requires that
its input array is in increasing order.
The index~\inl{lb}, that \lowerbound returns satisfies the inequality

\begin{align}
\Label{eq:lower-bound-result}
0 \leq \mathtt{lb} \leq n  
\end{align}

and has the following properties for a valid index~\inl{k} of the array under consideration

\begin{alignat}{3}
\Label{eq:lower-bound-left}
0 &\leq k < \mathtt{lb} && \qquad\Longrightarrow\qquad && a[k] < \mathtt{v} \\
\Label{eq:lower-bound-right}
\mathtt{lb} &\leq k < n && \qquad\Longrightarrow\qquad && \mathtt{v} \leq a[k]
\end{alignat}

Conditions~\eqref{eq:lower-bound-left} and~\eqref{eq:lower-bound-right} imply that~\inl{v}
can only occur in the array section \inl{a[lb..n-1]}.
In this sense \lowerbound returns a \emph{lower bound} for the potential indices.

As an example, we consider in Figure~\ref{fig:lowerbound} an increasingly ordered array.
The arrows indicate which indices will be returned by \lowerbound for a given value.
Note that the index~9 points \emph{one past end} of the array.
Values that are not contained in the array are colored in gray.

\begin{figure}[hbt]
\centering
\includegraphics[width=0.60\textwidth]{Figures/lower_bound.pdf}
\caption{\Label{fig:lowerbound}Some examples for \lowerbound}
\end{figure}

%\FloatBarrier

Figure~\ref{fig:lowerbound} also clarifies that care must
be taken when interpreting the return value of \lowerbound.
%
An important difference to the algorithms in Chapter~\ref{cha:non-mutating}
is that a return value of \lowerbound that is less than~$n$ 
does not necessarily implies \inl{a[lb] == v}.
We can only be sure that \inl{v <= a[lb]} holds.


\subsection{Formal specification of \lowerbound}

The specification of \specref{lowerbound} is shown in the following listing.
The preconditions \inl{increasing} expresses
that the array values need to be in increasing order.
%
The postconditions reflect the conditions listed above and can be expressed
using the predicates \logicref{LowerBound} and \logicref{StrictUpperBound}.

\begin{itemize}
\item Condition~\eqref{eq:lower-bound-result} becomes postcondition \inl{result}
\item Condition~\eqref{eq:lower-bound-left} becomes postcondition \inl{left}
\item Condition~\eqref{eq:lower-bound-right} becomes postcondition \inl{right}
\end{itemize}

\input{Listings/lower_bound.h.tex}

\subsection{Implementation of \lowerbound}
\Label{subsec:lowerbound:impl}

The following listing shows our implementation of \implref{lowerbound}.
Each iteration step narrows down the range that contains the
sought-after result. 
The loop invariants express that in each iteration step all indices
less than the temporary left bound
\inl{left} contain values that are less than \inl{v} and all indices
not less than the temporary right bound \inl{right} contain values
that are greater or equal than \inl{v}.
%
The expression to compute \inl{middle} is slightly more complex than the
naïve \inl{(left+right)/2}, but it avoids potential overflows.

\input{Listings/lower_bound.c.tex}

