
\chapter{Binary search algorithms}
\Label{cha:binary-search}

In this chapter, we consider the four
\emph{binary search} algorithms of the \cxx Standard Library \cite[\S
28.7.3]{cxx-17-draft}, namely

\begin{itemize}
\item \lowerbound in \S\ref{sec:lowerbound}

\item \upperbound in \S\ref{sec:upperbound}

\item two variants for the implementation of  \equalrange in \S\ref{sec:equalrange}

\item two variants for the formal specification of \binarysearch in \S\ref{sec:binarysearch}
\end{itemize}

As in the case of the of maximum/minimum algorithms from Chapter~\ref{cha:maxmin}
the binary search algorithms primarily use the less-than operator~\inl{<}
(and the derived operators \inl{<=}, \inl{>} and \inl{>=}) to determine whether a particular
value is contained in an increasing range.
Thus, different to the \find algorithm in \S\ref{sec:find},
the equality operator~\inl{==} will play only a supporting part
in the specification of binary search.

In order to make the specifications of the binary search algorithms 
more compact and (arguably) more readable we re-use the predicates
\logicref{LowerBound}, \logicref{StrictLowerBound},
\logicref{UpperBound}, and \logicref{StrictUpperBound}.

All binary search algorithms require that their input array is arranged in 
increasing order.
The following listing shows two versions of predicate \logicref{Increasing}.
The first one defines when a section of an array is in increasing order.
The second version uses the first one to express that the whole array is in increasing order.

\input{Listings/Increasing.acsl.tex}

%\clearpage

There is also the overloaded predicate \logicref{WeaklyIncreasing} that we will user for 
the verification of other algorithms.

\input{Listings/WeaklyIncreasing.acsl.tex}

Users inexperienced in formal verification often have a blind spot at the
difference between \Increasing and \WeaklyIncreasing.
%
Both versions are logically equivalent,
and proving that \Increasing implies \WeaklyIncreasing is even trivial.
%
However, proving the converse direction is not, and requires an induction on the
array size~\inl{n}, employing the transitivity of \inl{<=} in the induction step.
%
Humans are trained to perform such inductions unnoticed,
but none of the automated provers supported by \framac is able to perform induction.
%
The following Listing contains several lemmas on the relationship of
\WeaklyIncreasing and \Increasing.

\input{Listings/IncreasingLemmas.acsl.tex}

\clearpage

We usually exploit the relationship of the predicates \Increasing and \WeaklyIncreasing
in the following way:

\begin{itemize}
\item We use the predicate \Increasing in the preconditions and postconditions of
      function contracts.
\item The \WeaklyIncreasing is employed for assertions and loop invariants
      whenever we have to verify that an algorithm (typically a sorting algorithm)
      produces an increasing array.
\item Finally, to conclude that a \emph{weakly increasing} array is in fact \emph{increasing}
      we rely on lemma\\
      \logicref{WeaklyIncreasingIncreasing} .
\end{itemize}

\clearpage

\lowerbound & \ref{sec:lowerbound} & 22   &  22   & 100  &  10  &  12  &   0  &   0  &   0 \\\hline

\upperbound & \ref{sec:upperbound} & 36   &  36   & 100  &  18    &  18    &   0    &   0 \\\hline
\equalrange & \ref{sec:equalrange} & 22   &  22   & 100  &  17    &   5    &   0    &   0 \\\hline
\binarysearch & \ref{sec:binarysearch} & 10   &  10   & 100  &   8  &   0  &   2  &   0  &   0 \\\hline


