\section{The \equalrange algorithm}
\Label{sec:equalrange}
\Label{sec:equalrangeii}

The \equalrange algorithm is one of the four binary search algorithms
of the \cxx Standard Library \cite[\S 28.7.3.3]{cxx-17-draft}.
As with the other binary search algorithms \equalrange requires that
its input array is in increasing order.
The specification of \equalrange states that it \emph{combines} the results of the algorithms
\specref{lowerbound} and \specref{upperbound}.

For our purposes we have modified
\equalrange
to take an array of type \valuetype.
Moreover, instead of a pair of iterators, our version returns a pair of indices.
To be more precise, the return type of \equalrange is the
struct \sizetypepair from Listing~\ref{lst:size_type_pair}.
Thus, the signature of \equalrange now reads:

\begin{lstlisting}[style = acsl-block]

  size_type_pair
  equal_range(const value_type* a, size_type n, value_type v);
\end{lstlisting}

Figure~\ref{fig:equalrange} combines Figure~\ref{fig:lowerbound} with Figure~\ref{fig:upperbound}
in order visualize the behavior of \equalrange for select test cases.
The two types of arrows~$\rightarrow$ and~$\dashrightarrow$ represent the
respective fields \inl{first} and \inl{second} of the return value.
For values that are not contained in the array, the two arrows point to the same index.
More generally, if \equalrange returns the pair $(\mathtt{lb},\mathtt{ub})$, then
the difference $\mathtt{ub} - \mathtt{lb}$ is equal to the number of occurrences of the 
argument \inl{v} in the array.

\begin{figure}[hbt]
\centering
\includegraphics[width=0.60\textwidth]{Figures/equal_range.pdf}
\caption{\Label{fig:equalrange}Some examples for \equalrange}
\end{figure}

\FloatBarrier

We will provide two implementations of \equalrange and verify both of them.
The first implementation \implref{equalrange} just straightforwardly
calls \specref{lowerbound} and \specref{upperbound} and simply
returns the pair of their respective results.
The second implementation \implref{equalrangeii}, which is more elaborate, follows the
original \cxx code by attempting to minimize duplicate computations.
%
Let $(\mathtt{lb}, \mathtt{ub})$ be the return value  \equalrange, then
the conditions~\eqref{eq:lower-bound-result}--\eqref{eq:upper-bound-right} can
be merged into the inequality
%
\begin{align}
\Label{eq:equal-range-result}
0 \leq \mathtt{lb} \leq \mathtt{ub} \leq n 
\end{align}

and the following three implications for a valid index $k$ of the array under
consideration
%
\begin{alignat}{3}
\Label{eq:equal-range-left}
0           &\leq  k < \mathtt{lb} && \qquad\Longrightarrow\qquad &&  a[k] < \mathtt{v} \\
\Label{eq:equal-range-middle}
\mathtt{lb} &\leq k < \mathtt{ub} && \qquad\Longrightarrow\qquad &&  a[k] = \mathtt{v} \\
\Label{eq:equal-range-right}
\mathtt{ub} &\leq k < n           && \qquad\Longrightarrow\qquad &&  a[k] > \mathtt{v} 
\end{alignat}

%\clearpage

Here are some justifications for these conditions.

\begin{itemize}
\item 
Conditions~\eqref{eq:equal-range-left} and~\eqref{eq:equal-range-right} are just the 
Conditions~\eqref{eq:lower-bound-left} and~\eqref{eq:upper-bound-right}, respectively.

\item
The Inequality~\eqref{eq:equal-range-result} follows from the Inequalities~\eqref{eq:lower-bound-result}
and~\eqref{eq:upper-bound-result} and the following considerations:
If $\mathtt{ub}$ were less than $\mathtt{lb}$, then according to~\eqref{eq:equal-range-left}
we would have $a[\mathtt{ub}] < \mathtt{v}$.
One the other hand, we know from~\eqref{eq:equal-range-right} that opposite
inequality $\mathtt{v} < a[\mathtt{ub}]$ holds.
Therefore, we have $\mathtt{lb} \leq \mathtt{ub}$.

\item
Condition~\eqref{eq:equal-range-middle} follows from the combination of~\eqref{eq:lower-bound-right}
and~\eqref{eq:upper-bound-left} and the fact that~$\leq$ is a total order on the integers.
\end{itemize}

%\clearpage

\subsection{Formal specification of \equalrange}

The following listing show the specification of \specref{equalrange}
(and of \equalrangeii).

\input{Listings/equal_range.h.tex}

The precondition \inl{increasing} expresses
that the array values need to be in increasing order.

The postconditions reflect the conditions listed above and can be expressed
using the already introduced predicates
\logicref{AllEqual},
\logicref{StrictUpperBound} and \logicref{StrictLowerBound}.

\begin{itemize}
\item Condition~\eqref{eq:equal-range-result} becomes postcondition \inl{result}
\item Condition~\eqref{eq:equal-range-left} becomes postcondition \inl{left}
\item Condition~\eqref{eq:equal-range-middle} becomes postcondition \inl{middle}
\item Condition~\eqref{eq:equal-range-right} becomes postcondition \inl{right}
\end{itemize}

%\clearpage 

\subsection{Implementation of \equalrange}

Our first implementation of \implref{equalrange} is shown in the following listing.
We just call the two functions \specref{lowerbound} and \specref{upperbound}
and return their respective results as a pair.

\input{Listings/equal_range.c.tex}

In a very early version of this document we had proven the similar assertion
\inl{first <= second} with the interactive theorem prover \coq.
After reviewing this proof we formulated the new assertion \inl{aux}
that uses a fact from the postcondition of \specref{upperbound}.
The benefit of this reformulation is that both the assertion \inl{aux}
and the postcondition \inl{first <= second} can now be verified automatically.

%\clearpage 

\subsection{Implementation of \equalrangeii}

The first implementation of \implref{equalrange} does more work than needed.
In the following listing \implref{equalrangeii} we show that it is possible to
perform as much range reduction as possible before calling \specref{upperbound}
and \specref{lowerbound} on the reduced ranges.


\input{Listings/equal_range2.c.tex}

Due to the higher code complexity of the second implementation, additional
assertions had to be inserted in order to ensure that \wpframac is able to verify the
correctness of the code.
All of these assertions are related to pointer arithmetic and shifting base pointers.
They fall into three groups and are briefly discussed below.
In order to enable the automatic verification of these properties we
added the following collection of \logicref{ArrayBoundsShift}.

\input{Listings/ShiftLemmas.acsl.tex}

\subsubsection*{The \inl{increasing} properties}

Both \specref{lowerbound} and \specref{upperbound} expect that they
operate on increasingly ordered arrays.
This is of course also true for \specref{equalrange}, however,
inside our second implementation we need a more specific formulation, namely,

\begin{lstlisting}[style=acsl-block]

        Increasing(a + middle, last - middle)
\end{lstlisting}

With the three-argument form of predicate \logicref{Increasing}
we can formulate out an intermediate step.
This enables the provers to verify the preconditions of the call to
\specref{lowerbound} automatically.
A similar assertion is present before the call to \specref{upperbound}.

%\clearpage

\subsubsection*{The \inl{strict} and \inl{constant} properties}

Part of the post conditions of \specref{equalrange} is that \inl{v}
is both a strict upper and a strict lower bound.
However, the calls to \lowerbound and \upperbound only give us

\begin{lstlisting}[style=acsl-block]

        StrictUpperBound(a + first, 0, left - first, v) 

        StrictLowerBound(a + middle, right - middle, last - middle, v)
\end{lstlisting}

which is not enough to reach the desired post conditions automatically.
One intermediate step for each of the assertions was sufficient to guide
the prover to the desired result.

Conceptually similar to the \inl{strict} properties
the \inl{constant} properties guide the prover towards

\begin{lstlisting}[style=acsl-block]

        LowerBound(a, left, n, v) 

        UpperBound(a, 0, right, v)
\end{lstlisting}

Combining these properties allow the postcondition \inl{middle} to be derived automatically.

%\clearpage

