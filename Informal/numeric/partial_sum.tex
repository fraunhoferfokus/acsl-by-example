
\section{The \partialsum algorithm}
\Label{sec:partialsum}

The \partialsum algorithm in the \cxx Standard Library \cite[\S 29.8.6]{cxx-17-draft} computes
the sum of a given initial value and the elements in a range.
%
Our version of the original signature
reads:

\begin{lstlisting}[style=acsl-block]

  size_type
  partial_sum(const value_type* a, size_type n, value_type* b);
\end{lstlisting} 

After executing the function \partialsum the array \inl{b[0..n-1]} holds the following values
\begin{align}
\Label{eq:partialsum}
   \mathtt{b}[i] &= \sum_{k = 0}^{\mathtt{i}} \mathtt{a}[k]
\end{align}
for $0 \leq i < n$.
%
Equations~\eqref{eq:partialsum} and~\eqref{eq:accumulate-default}
suggest that we define in the following listing the \acsl predicate \PartialSum
by using the logic function \logicref{AccumulateDefault}.

\input{Listings/PartialSum.acsl.tex}

\clearpage

\subsection{Formal specification of \partialsum}

The specification of \specref{partialsum} demands that the arrays
\inl{a[0..n-1]} and \inl{b[0..n-1]} 
are separated, that is, they do not overlap.
Note that is a stricter requirement than in the case of the original
\cxx version of \partialsum, which allows that~\inl{a} equals~\inl{b},
thus allowing the computation of partial sums \emph{in place}.

\input{Listings/partial_sum.h.tex}

\clearpage

\subsection{Implementation of \partialsum}

The following listing shows an implementation of \partialsum with corresponding loop annotations.

\input{Listings/partial_sum.c.tex}

\clearpage

