
\section{The \accumulate algorithm}
\Label{sec:accumulate}

The \accumulate algorithm in the \cxx Standard Library \cite[\S 29.8.2]{cxx-17-draft} computes
the sum of an given initial value and the elements in a range.
%
Our version of the original signature
reads:

\begin{lstlisting}[style=acsl-block]

  value_type
  accumulate(const value_type* a, size_type n, value_type init);
\end{lstlisting} 

The result of \accumulate shall equal the value
$\displaystyle{ \mathtt{init} + \sum_{i = 0}^{\mathtt{n}-1} \mathtt{a}[i] }$.
This implies that \accumulate will return \inl{init} for an empty range.

%\clearpage

\subsection{The logic function \Accumulate}

As in the case of \specref{counti}  we specify \accumulate by first defining the
\emph{logic function} \logicref{Accumulate} that formally defines
the summation of elements in an array.

\input{Listings/Accumulate.acsl.tex}

With this definition the following equation holds for $n \geq 0$
\begin{align}
\Label{eq:accumulate}
    \mathtt{Accumulate}(\mathtt{a}, \mathtt{n}, \mathtt{init})
    &= \mathtt{init} + \sum_{i = 0}^{\mathtt{n-1}} \mathtt{a}[i]
\end{align}

The predicate \logicref{AccumulateBounds} that we will subsequently use
in order to compactly express requirements that exclude numeric
overflows while accumulating value.
This predicate states that  for $0 \leq i < n$ the \emph{partial sums} 
%
\begin{gather}
\Label{eq:accumulate1}
\mathtt{init} + \sum_{k = 0}^{\mathtt{i}} \mathtt{a}[k]
\end{gather}
%
do not overflow.
If one of them did, one couldn't guarantee that the result of \isoc implementation
of \accumulate equals the mathematical description of \Accumulate.

%\clearpage

\subsection{\AccumulateDefault ---a variant of \Accumulate}

The following listing shows another version of \logicref{Accumulate},
called \logicref{AccumulateDefault}.

\input{Listings/AccumulateDefault.acsl.tex}

The function \AccumulateDefault uses~\inl{a[0]} as default value of \inl{init}.
Thus, for \AccumulateDefault we have

\begin{align}
\Label{eq:accumulate-default}
    \mathtt{AccumulateDefault}(\mathtt{a}, \mathtt{n})
    &= \sum_{i = 0}^{\mathtt{n-1}} \mathtt{a}[i]
\end{align}
We will use this version for the specification of the algorithm \specref{partialsum}.

This listing also includes additional properties of observable
\AccumulateDefault behavior, here given as a lemmas.
It also contains the predicate \logicref{AccumulateDefaultBounds}
with corresponding numeric limits for the predicate~\AccumulateDefault.

%\clearpage

\subsection{Formal specification of \accumulate}

Using the logic function \Accumulate and the predicate \AccumulateBounds,
the specification of \accumulate is then as simple
as shown in the following listing.

\input{Listings/accumulate.h.tex}

\clearpage

\subsection{Implementation of \accumulate}

The following listing shows an implementation of the
\accumulate function with corresponding loop annotations.

\input{Listings/accumulate.c.tex}

Note that loop invariant \inl{partial} claims that in the $i$-th iteration step \inl{result}
equals the accumulated value of Equation~\eqref{eq:accumulate1}.
This depends on the property \inl{bounds} of \specref{accumulate} which expresses that
there is no numeric overflow when updating the variable \inl{init}.

\clearpage
