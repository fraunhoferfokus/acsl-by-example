
\chapter{Numeric algorithms}
\Label{cha:numeric}

The algorithms that we considered so far only \emph{compared}, \emph{read} or
\emph{copied} values in sequences.
In this chapter, we consider so-called \emph{numeric} algorithms of the 
\cxx Standard Library \cite[\S 29.8]{cxx-17-draft} that use arithmetic
operations on \valuetype to
combine the elements of sequences.

\begin{listing}[hbt]
\begin{center}
\begin{minipage}{0.5\textwidth}
\begin{lstlisting}[style=acsl-block]
    #define VALUE_TYPE_MAX  INT_MAX 
    #define VALUE_TYPE_MIN  INT_MIN
\end{lstlisting}
\end{minipage}
\end{center}
\vspace*{-0.5cm}
\caption{\Label{lst:value-type-limits}Limits of \valuetype}
\end{listing}

In order to refer to potential arithmetic overflows we introduce the
two constants shown in Listing~\ref{lst:value-type-limits}
which refer to the numeric limits of \valuetype 
(see also \S\ref{sec:types}).

We consider the following algorithms.

\begin{itemize}

\item \iotai 
writes sequentially increasing values into a range
(\S\ref{sec:iotai})
 
\item \accumulate 
computes the sum of the elements in a range
(\S\ref{sec:accumulate})

\item \innerproduct 
computes the inner product of two ranges
(\S\ref{sec:innerproduct})

\item \partialsum 
computes the sequence of partial sums of a range
(\S\ref{sec:partialsum})

\item \adjacentdifference 
computes the differences of adjacent elements in a range 
(\S\ref{sec:adjacentdifference})

\item
Finally, in \S\ref{sec:partialsuminv} we show that under
appropriate preconditions the algorithms \partialsum and
\adjacentdifference are inverse to each other.

\end{itemize}

The formal specifications of these algorithms raise new questions.
In particular, we now have to deal with arithmetic overflows in \valuetype.

\clearpage

\IOTA & \ref{sec:iota} & 18   &  18   & 100  &   7  &   0  &  11  &   0  &   0 \\\hline

\accumulate & \ref{sec:accumulate} & 18   &  18   & 100  &  10    &   8    &   0    &   0 \\\hline
\innerproduct & \ref{sec:innerproduct} & 23   &  23   & 100  &  10    &  13    &   0    &   0 \\\hline
\partialsum & \ref{sec:partialsum} & 41   &  41   & 100  &   8  &  31  &   1  &   1  &   0 \\\hline

\adjacentdifference & \ref{sec:adjacentdifference} & 35   &  35   & 100  &  10  &   0  &  25  &   0  &   0 \\\hline


\section{Inverting \partialsum and \adjacentdifference}
\Label{sec:partialsuminv}
\Label{sec:adjacentdifferenceinv}


In this section we show that under appropriate preconditions
the algorithms \partialsum and\\
\adjacentdifference are inverse to each other.

\subsection{Inverting \partialsum}

Let \inl{a[0..n-1]} and \inl{b[0..n-1]} be the respective input and output 
of \partialsum.
We have in other words
\begin{align*}
   \mathtt{b}[0] &= \mathtt{a}[0] \\
   \mathtt{b}[1] &= \mathtt{a}[0] + \mathtt{a}[1] \\
                 &\vdotswithin{=} \\
   \mathtt{b}[n-1]  &= \mathtt{a}[0] + \mathtt{a}[1] + \ldots + \mathtt{a}[n-1] \\
\end{align*}

If we apply now the algorithm \adjacentdifference to \inl{b[0..n-1]}, then
we find for its output \inl{a'[0..n-1]}
\begin{alignat*}{3}
   \mathtt{a'}[0] &= \mathtt{b}[0]                        &\quad &=\quad \mathtt{a}[0] \\
   \mathtt{a'}[1] &= \mathtt{b}[1] - \mathtt{b}[0]        &\quad &=\quad \mathtt{a}[1] \\
                 &\vdotswithin{=}  \\
   \mathtt{a'}[n-1] &= \mathtt{b}[n-1] - \mathtt{b}[n-2]  &\quad &=\quad \mathtt{a}[n-1]
\end{alignat*}


Before we start show the \acsl lemmas of our claim, we present
the predicate \logicref{DefaultBounds} in order to express that the values
in the input (and output!) array~\inl{a[0..n-1]} do not overflow.

\input{Listings/DefaultBounds.acsl.tex}

Lemma \PartialSumInverse from the following listing
expresses as \acsl lemmas
that the algorithms \partialsum and \adjacentdifference are inverse to each other.

\input{Listings/NumericInverse.acsl.tex}

%\clearpage

The following listing now shows \isoc function \partialsuminv
(both the contract and the implementation).
This function calls first \partialsum and then \adjacentdifference.

\input{Listings/partial_sum_inv.c.tex}

The contract of \partialsuminv formulates preconditions that shall guarantee
that during the computation neither arithmetic overflows (property~\inl{bounds})
nor unintended aliasing of arrays (property~\inl{sep}) occur.
Under these precondition, \framac shall verify
that the final call to \specref{adjacentdifference} just restores the original contents
of~\inl{a[0..n-1]} that we supplied for the initial call to \specref{partialsum}.

\clearpage

\subsection{Inverting \adjacentdifference}

After executing the function \specref{adjacentdifference} on
the input array \inl{a[0..n-1]} the output array \inl{b[0..n-1]}
holds the following values
\begin{align*}
   \mathtt{b}[0] &= \mathtt{a}[0] \\
   \mathtt{b}[1] &= \mathtt{a}[1] - \mathtt{a}[0] \\
                 &\vdotswithin{=} \\
   \mathtt{b}[n-1] &= \mathtt{a}[n-1] - \mathtt{a}[n-2] \\
\end{align*}

If we call now \partialsum with the array \inl{b[0..n-1]}
as input, then we obtain for its output \inl{a'[0..n-1]}
\begin{alignat*}{3}
   \mathtt{a'}[0] &= \mathtt{b}[0]
                  &\quad &=\quad \mathtt{a}[0] \\
   \mathtt{a'}[1] &= \mathtt{b}[0] + \mathtt{b}[1]
                  &\quad &=\quad \mathtt{a}[1] \\
                  &\vdotswithin{=} \\
   \mathtt{a'}[n-1]  &=\ \mathtt{b}[0] + \mathtt{b}[1] + \ldots + \mathtt{b}[n-1]
                     &\quad &=\quad \mathtt{a}[n-1] 
\end{alignat*}
%
which means that applying \specref{partialsum} on the output of
\adjacentdifference produces the original input.
Lemma \logicref{AdjacentDifferenceInverse} expresses this property as a lemma.

The function \implref{adjacentdifferenceinv} first calls \adjacentdifference and then\\
\partialsum.
The contract of this function formulates preconditions that shall guarantee
that during the computation neither arithmetic overflows (property~\inl{bound})
nor unintended aliasing of arrays (property~\inl{sep}) occur.
In order to improve the automatic verification of 
\adjacentdifferenceinv we also use lemma \logicref{UnchangedTransitive}.
Lemma \logicref{AdjacentDifferenceInverseBounds} simplifies
the verification of the precondition \inl{bounds} of \partialsum.

\input{Listings/adjacent_difference_inv.c.tex}



