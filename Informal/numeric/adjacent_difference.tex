
\section{The \adjacentdifference algorithm}
\Label{sec:adjacentdifference}

The \adjacentdifference algorithm in the \cxx Standard Library \cite[\S 29.8.11]{cxx-17-draft}
computes the differences of adjacent elements in a range.
%
Our version of the original signature reads:

\begin{lstlisting}[style=acsl-block]

size_type
adjacent_difference(const value_type* a, size_type n, value_type* b);
\end{lstlisting} 

After executing the function \adjacentdifference the array \inl{b[0..n-1]} holds the following values
\begin{align}
   \mathtt{b}[0] &= \mathtt{a}[0] \nonumber \\
   \mathtt{b}[1] &= \mathtt{a}[1] - \mathtt{a}[0] \nonumber \\
                 &\vdotswithin{=} \nonumber \\
   \mathtt{b}[n-1] &= \mathtt{a}[n-1] - \mathtt{a}[n-2] \nonumber \\
\end{align}

\subsection{The predicate \AdjacentDifference}

We start with the definition of the logic function \Difference whose definition
is shown in the following listing.

\input{Listings/Difference.acsl.tex}

\clearpage 

Building on top of \Difference we now introduce the predicate \AdjacentDifference.
We also provide the predicate \AdjacentDifferenceBounds
that captures conditions that prevent numeric overflows
while computing differences of the form \inl{a[i] - a[i-1]}.

\input{Listings/AdjacentDifference.acsl.tex}

Lemmas \logicref{AdjacentDifferenceStep} and \logicref{AdjacentDifferenceSection}
will help us later in the verification of \implref{adjacentdifferenceinv}.

\clearpage

\subsection{Formal specification of \adjacentdifference}

Using the predicates \logicref{AdjacentDifference} and \logicref{AdjacentDifferenceBounds}
we can provide in the following listing a concise formal specification of \adjacentdifference.
As in the case of the specification of \specref{partialsum}
we require that the arrays \inl{a[0..n-1]} and \inl{b[0..n-1]} 
are separated.

\input{Listings/adjacent_difference.h.tex}

\clearpage

\subsection{Implementation of \adjacentdifference}

The following listing shows an implementation of \adjacentdifference
with corresponding loop annotations.
In order to achieve the verification of the loop invariant \inl{difference} we 
rely on
\begin{itemize}
\item the assertions \inl{bound} and \inl{difference}
\item the lemmas \logicref{AdjacentDifferenceStep} and \logicref{AdjacentDifferenceSection}
\item a statement contract with the two postconditions labeled as \inl{step}
\end{itemize}

\input{Listings/adjacent_difference.c.tex}

\clearpage

