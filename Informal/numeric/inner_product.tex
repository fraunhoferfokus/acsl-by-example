
\section{The \innerproduct algorithm}
\Label{sec:innerproduct}

The \innerproduct algorithm in the \cxx Standard Library \cite[\S 29.8.4]{cxx-17-draft} computes
the \emph{inner product}\footnote{
  Also referred to as \emph{dot product}, see \url{http://en.wikipedia.org/wiki/Dot_product}
}
of two ranges.
%
Our version of the original signature
reads:

\begin{lstlisting}[style=acsl-block]

  value_type
  inner_product(const value_type* a, const value_type* b,
                size_type n, value_type init);
\end{lstlisting} 

The result of \innerproduct equals the value
\begin{gather*}
\mathtt{init} + \sum_{i = 0}^{\mathtt{n}-1} \mathtt{a}[i] \cdot \mathtt{b}[i]
\end{gather*}
thus, \innerproduct will return \inl{init} for empty ranges.

%\clearpage

\subsection{The logic function \InnerProduct}

As in the case of \specref{accumulate} we specify \innerproduct
by defining in the following listing the logic function \InnerProduct
that formally expresses the summation of the element-wise product of two arrays.

Predicate \logicref{ProductBounds} expresses that for $0 \leq i < n$ the products 
%
\begin{align}
\Label{eq:innerproduct1}
\mathtt{a}[i] \cdot \mathtt{b}[i] 
\end{align}
%
do not overflow. 
Predicate \logicref{InnerProductBounds}, on the other hand, states that for $0 \leq i < n$
the following sums do not overflow.
cc%
\begin{align}
\Label{eq:innerproduct2}
\mathtt{init} + \sum_{k = 0}^{\mathtt{i}} \mathtt{a}[k] \cdot \mathtt{b}[k]
\end{align}

Otherwise, one cannot guarantee that the result of our implementation
of \implref{innerproduct} equals the mathematical description of \InnerProduct.
Finally, Lemma \logicref{InnerProductUnchanged} states that the result of the \InnerProduct only
depends on the values of \inl{a[0..n-1]} and \inl{b[0..n-1]}.

\input{Listings/InnerProduct.acsl.tex}

\subsection{Formal specification of \innerproduct}

Using the logic function \logicref{InnerProduct}, we specify \innerproduct as shown 
in the following listing.
Note that we needn't require that \inl{a} and \inl{b} are separated.

\input{Listings/inner_product.h.tex}

\clearpage

\subsection{Implementation of \innerproduct}

The following listing shows an implementation of \innerproduct
with corresponding loop annotations.

\input{Listings/inner_product.c.tex}

Note that the loop invariant \inl{inner} claims
that in the $i$-th iteration step the current value of \inl{init}
equals the accumulated value of Equation~\eqref{eq:innerproduct2}.
This depends of course on the properties \inl{bounds} in the contract
of \specref{innerproduct}, which express that there is no arithmetic overflow
when computing the updates of the variable \inl{init}.

\clearpage

