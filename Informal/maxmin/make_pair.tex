
\section{The auxiliary function \makepair}
\label{sec:makepair}

In order to be able to specify functions that work
on pairs of indices we introduce in the following listing
the type \sizetypepair.

\begin{listing}[hbt]
\begin{minipage}{0.99\textwidth}
\lstinputlisting[style=acsl-block, frame=single]{Source/size_type_pair.h}
\end{minipage}
  \caption{\Label{lst:size_type_pair}The type \sizetypepair}
\end{listing}

\index[examples]{size\_pair\_type@\texttt{size\_pair\_type}}

%\clearpage

We will also use the auxiliary function \makepair which turns
two indices \inl{first} and \inl{second} into an object of \sizetypepair.
The specification and implementation of \specref{makepair} is shown here.

\input{Listings/make_pair.h.tex}

\clearpage

