
\section{The \maxelement algorithm}
\Label{sec:maxelement}

The \maxelement algorithm in the \cxx Standard Library \cite[\S 28.7.8]{cxx-17-draft}
searches the maximum of a general sequence. 
The signature of our version of \maxelement reads:

\begin{lstlisting}[style = acsl-block]

  size_type max_element(const value_type* a, size_type n);
\end{lstlisting}

The function finds the largest element in the range
\inl{a[0..n-1]}.
More precisely, it returns the unique valid index \inl{i} such that:
\begin{enumerate}
\item for each index \inl{k} with \inl{0 <= k < n} the condition
\inl{a[k] <= a[i]} holds and
\item for each index \inl{k} with \inl{0 <= k < i} the condition
\inl{a[k] < a[i]} holds.
\end{enumerate}
The return value of \maxelement is \inl{n} if and only if there is no
maximum, which can only occur if \inl{n == 0}.

\subsection{Formal specification of \maxelement}

The following listings shows the formal specification of \specref{maxelement}.
Note that we have subdivided the specification of \maxelement into the two
behaviors~\inl{empty} and \inl{not_empty}.
The behavior \inl{empty} contains the specification for the case that the
range contains no elements.
The behavior \inl{not_empty} applies if the range has a positive length.

The ensures clause \inl{max} of behavior \inl{not_empty} indicates that
the returned valid index \inl{k} refers to a maximum value of the array.
The postcondition \inl{first} expresses that \inl{k} is indeed the
\emph{first} occurrence of a maximum value in the array.

\input{Listings/max_element.h.tex}

\subsection{Implementation of \maxelement}

In our description, we concentrate on the \emph{loop annotations}
of the implementation of \implref{maxelement}.

\input{Listings/max_element.c.tex}

The loop invariant \inl{max} is needed to prove the  postcondition
\inl{result} of the behavior \inl{not_empty} of \specref{maxelement}.
Using loop invariant \inl{upper} we prove the postcondition \inl{upper}
of the behavior \inl{not_empty} of \specref{maxelement}.
Finally, the postcondition \inl{first} of this behavior can be
verified with the loop invariant \inl{first}.

\clearpage

