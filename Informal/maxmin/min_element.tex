
\section{The \minelement algorithm}
\Label{sec:minelement}

The \minelement  algorithm in the \cxx Standard Library \cite[\S 28.7.8]{cxx-17-draft}
searches the minimum in a general sequence. 
The signature of our version of \minelement reads:

\begin{lstlisting}[style = acsl-block]

  size_type min_element(const value_type* a, size_type n);
\end{lstlisting}

The function \minelement finds the smallest element in the range \inl{a[0..n-1]}.
More precisely, it returns the unique valid index \inl{i} such that
\inl{a[i]} is minimal among the values \inl{a[0]}, \ldots,
\inl{a[n-1]}, and \inl{i} is the first position with that property.
The return value of \minelement is \inl{n} if and only if \inl{n == 0}.

We use the predicate \logicref{LowerBound} that
basically expresses that a given value is less or equal than all
elements of a given array (section).
%
Closely related to the predicate \LowerBound is the predicate \logicref{StrictLowerBound}.
%
We also use the predicate \logicref{MinElement} which states that the element
at a given index \inl{min} is a \emph{lower bound} of the sequence \inl{a[0..n-1]},
and, by construction, a member of that sequence.


\subsection{Formal specification of \minelement}


The following listing contains the specification of \specref{minelement}.
Note that we also use the predicate \logicref{StrictLowerBound} in order to
express that \minelement returns the \emph{first} minimum position in \inl{a[0..n-1]}.

\input{Listings/min_element.h.tex}

\clearpage

\subsection{Implementation of \minelement}

The implementation of \implref{minelement} uses the predicates \logicref{LowerBound}
and \logicref{StrictLowerBound} in its loop annotations.

\input{Listings/min_element.c.tex}

\clearpage

