
\section{The \maxelement algorithm with predicates}
\Label{sec:maxelementii}

In this section we present another specification of the \maxelement algorithm.
The main difference is that we employ the predicate \logicref{UpperBound}
which basically expresses that a given value is greater or equal than all
elements of a given array.
Closely related to the predicate \UpperBound is the predicate \logicref{StrictUpperBound}.

We also employ the predicate \logicref{MaxElement}.
This predicate states that the element at a given index \inl{max} is an 
\emph{upper bound} of the sequence \inl{a[0..n-1]}, and, by
construction, a member of that sequence.

\subsection{Formal specification of \maxelementii}

The formal specification of \specref{maxelementii} is shown in the following listing.
Note that we also use the predicate  \logicref{StrictUpperBound}
in order to express that \maxelementii returns the \emph{first} maximum position in \inl{a[0..n-1]}.

\input{Listings/max_element2.h.tex}

\clearpage

\subsection{Implementation of \maxelementii}

The implementation of \implref{maxelementii} is of course
very similar to that of \implref{maxelement}---except that the
loop invariants now also use the above mentioned predicates.

\input{Listings/max_element2.c.tex}

\clearpage

