
\section{The \partialsort algorithm}
\Label{sec:partialsort}

Our version of the \partialsort algorithm compared to the \cxx Standard
Library \cite[\S 28.7.1.3]{cxx-17-draft} has the signature

\begin{lstlisting}[style = acsl-block]

  void partial_sort(value_type* a, size_type m, size_type n);
\end{lstlisting}

The algorithm \emph{reorders} the given array \inl{a} in such a way
that it represents a \emph{partition}:
each member of the
left part \inl{a[0..m-1]} is less or equal to each member of the right
part \inl{a[m..n-1]}.
%
Moreover, the algorithm \emph{sorts} the left part in increasing order.
The order of elements in the right part, however, is \emph{unspecified}.
%
Figure~\ref{fig:partialsort} uses a bar chart to
depict a typical result of a call \inl{partial_sort(a, m, n)}.
%
In the post-state, 
the left and the right part is colored in green and orange,
respectively.


\begin{figure}[hbt]
\begin{center}
\includegraphics[width=0.75\textwidth]{Figures/partial_sort.pdf}
\caption{\Label{fig:partialsort} Effects of \partialsort}
\end{center}
\end{figure}

\FloatBarrier

\subsection{The predicate \Partition}

We start by introducing the new predicate \logicref{Partition}
which formalizes the partitioning property.

\input{Listings/Partition.acsl.tex}

\clearpage

The lemmas in the following listing are used in proofs of properties and annotations
related to the loop invariants \inl{upper}, \inl{lower}, and \inl{partition}
of \partialsort.

\input{Listings/PartitionLemmas.acsl.tex}


\begin{itemize}
\item
Lemma \MultisetReorderSomeEqual states that a value
\inl{a[i]} taken from a range \inl{a[0..n-1]} after some reordering
must have been in that range already before reordering.
It is used to prove the subsequent lemmas.

\item
Lemma \MultisetReorderLowerBound
informally says that a lower bound \inl{v} of a
range \inl{a[0..n-1]} keeps its property even after the range is
reordered.

\item
Dually, lemma \MultisetReorderUpperBound says that reordering a range
doesn't affect any of its upper bounds.

\item
Lemma \MultisetReorderPartitionLowerBound 
describes a more particular
situation: if each element in \inl{a[0..m-1]}
is known to be a less or equal than element \inl{a[m..n-1]}
and the former range is reordered while the latter is kept untouched,
then \inl{a[0]} will still be a lower bound of \inl{a[m..n-1]}.
We employ this lemma to infer that, after \specref{pushheap} was called, the new
heap maximum \inl{a[0]}, is a lower bound of \inl{a[m..i]},

\end{itemize}

The proof of \logicref{MultisetReorderSomeEqual} relies on the lemma \logicref{CountSomeEqual}.
We also rely on the lemma \logicref{MultisetSwapMiddle}
in order to verify that the loop invariant \inl{reorder} is preserved.

\subsection{Formal specification of \partialsort}

The formal specification of the \partialsort function is shown in the following listing.
It uses the just introduced predicate \Partition and reuses the
previously defined predicates \logicref{Increasing} and \logicref{MultisetReorder}.

\input{Listings/partial_sort.h.tex}

\subsection{Implementation of \partialsort}

Our implementation of \partialsort is shown the next listing.
%
It initially calls \specref{makeheap} to rearrange the left part \inl{a[0..m-1]} into a heap.
%
After that, it scans the right part, from left to right, for elements
that are too small;
each such element is exchanged for the left part's maximum, by applying
\specref{popheap}  and \specref{pushheap}  appropriately.
%
When the scan is done, the smallest elements are collected in the left
part.
%
We finally convert it from a heap into an increasingly ordered range,
by \sortheap (\ref{sec:sortheap}).

\begin{figure}[hbt]
\begin{center}
\includegraphics[width=0.50\textwidth]{Figures/partial_sort-loop.pdf}
\caption{\Label{fig:partialsort-loop}An iteration of \partialsort}
\end{center}
\end{figure}

\clearpage

In the scan loop, we maintain as invariants
\begin{itemize}
\item that the left part is a heap (invariant \inl{heap});
\item that its maximal element, \inl{a[0]}, is a ``separating element''
  between the left part \inl{a[0..m-1]} and the right part \inl{a[m..i-1]},
  i.e., an upper bound of the left (invariant \inl{upper})
  and a lower bound of the right part (invariant \inl{lower}), respectively;
\item that \inl{a[i..m-1]} is yet unchanged (invariant \inl{unchanged}); and
\item that only permutation operations have been applied to
  \inl{a[0..i-1]} (invariant \inl{reorder}).
\end{itemize}

In order to preserve the loop invariants after \inl{i} is incremented,
nothing has to be done if \inl{a[0]} happens to be
also a lower bound for \inl{a[i]}.
Otherwise, let us follow the algorithm through the \inl{then} part code,
depicting the intermediate states in 
Figure~\ref{fig:partialsort-loop}.
The elements considered so far are shown colored similar to
Figure~\ref{fig:partialsort}; in particular the heap part is shown in green.

%\clearpage

%
The overlaid transparent red shape indicates the ranges to which
\Partition applies, in each state.
%
The figure assumes the initial contents of \inl{a[0]} and
\inl{a[i]} to be $9$ and $5$,
for sake of generality, let us
call them $p$ and $q$, respectively.

After \popheap and \swap,
we have $p$ at \inl{a[i]}, and $q$ at \inl{a[m-1]}.
%
At that point we know
%
\begin{enumerate}
\item $q < p \leq \mbox{\inl{a[k]}}$ for each $m \leq k < i$,
  since $p$ was a lower bound for \inl{a[m..i-1]};
\item $q < p = \mbox{\inl{a[i]}}$;
\item $\mbox{\inl{a[j]}} \leq p \leq \mbox{\inl{a[k]}}$ 
  for each $0 \leq j < m-1$ and each $m \leq k < i$,
  since this held on loop entry, and we didn't more than
  reordering inside the parts; and
\item $\mbox{\inl{a[j]}} \leq p = \mbox{\inl{a[i]}}$ 
  since $p$ was the heap maximum on loop entry.
\end{enumerate}

\begin{listing}[t]
\begin{minipage}{\textwidth}
\lstinputlisting[linerange={1-38}, style=acsl-block, frame=single]{Source/partial_sort.c}
\end{minipage}
\caption{\Label{lst:partialsort-impl1}Implementation of \partialsort (1)}
\end{listing}

\index[examples]{partial\_sort@\texttt{partial\_sort}}


\FloatBarrier

Altogether, we have  $\mbox{\inl{a[j]}} \leq p \leq \mbox{\inl{a[k]}}$
for each $0 \leq j < m$ and each $m \leq k < i+1$.
%
That is, \inl{Partition(a,m,i+1)} holds, although we cannot name a
separating element of \inl{a} here.


After calling \pushheap, which just performs some more 
reorderings of the left part, this property is preserved. 
We can't and we needn't tell which position $q$ is moved to;
the former is indicated in Figure~\ref{fig:partialsort}
by the vague grey triangle.
%
Moreover, we now know again that \inl{a[0]} has become an upper bound
of the left part,
and hence a separating element between
\inl{a[0..m-1]} and \inl{a[m..i]};
that is, the loop invariants \inl{upper} and \inl{lower} have been
re-established.
%
These two invariants together are eventually used to prove
the property \inl{partition} of the contract.

Compared to its size, the algorithm makes a
lot of procedure calls; in this respect it is closer to real-life
software than most other algorithms of this tutorial.
%
Therefore, we use it to illustrate a methodical point:
%
For almost every procedure call, we give the callee's contract,
tailored to its actual parameters, as a statement contract of the call.
%
For example, everything we know from the \popheap contract,
instantiated to the particular situation, is documented in the
first statement contract.
%
In contrast, we use \inl{assert} clauses to indicate intermediate
reasoning to obtain subsequently needed properties.

\begin{listing}[t]
\begin{minipage}{\textwidth}
\lstinputlisting[linerange={39-99}, style=acsl-block, frame=single]{Source/partial_sort.c}
\end{minipage}
\caption{\Label{lst:partialsort-impl2}The Implementation of
\partialsort (2)}
\end{listing}

\index[examples]{partial\_sort@\texttt{partial\_sort}}

\FloatBarrier


Our implementation has a worst-case time complexity of
${\cal O}((n+m) \cdot \log m)$.
%
On the other hand, an implementation that ignores \inl{m} and just sorts \inl{a[0..n-1]}
also satisfies the contract of \specref{partialsort},
and may have ${\cal O}(n \cdot \log n)$ complexity.
%
Some arithmetic shows that \partialsort performs better than
plain sort if, and only if,
$\log m < \dfrac{n}{m} \cdot \log\left(\dfrac{n}{m}\right)$,
that is, if $n$ is sufficiently larger than $m$.



\clearpage

