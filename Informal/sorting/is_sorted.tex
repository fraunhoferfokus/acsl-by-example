
\section{The \issorted algorithm}
\Label{sec:issorted}

Our version of the \issorted algorithm compared to the \cxx Standard
Library \cite[\S 28.7.1.5]{cxx-17-draft} has the signature
\begin{lstlisting}[style = acsl-block]

  bool is_sorted(const value_type* a, size_type n);
\end{lstlisting}

It returns \inl{true} if the given array is in increasing order, and
\inl{false} otherwise.

\FloatBarrier

\subsection{Formal specification of \issorted}

The following listing shows the acsl specification of \issorted.
%
In the contract, we use the predicate \logicref{Increasing},
which states that any array element is always less or equal to any other element right of it.
%
We'll use an easier-to-handle predicate in the implementation of \implref{issorted}.

\input{Listings/is_sorted.h.tex}

\clearpage

\subsection{Implementation of \issorted}

The implementation of \issorted is shown in the next Listing.
%
As usual, \issorted doesn't compare every array element to all that are right
to it, but only to the immediately adjacent one, which is of course
more efficient.
For this, we use the predicate \logicref{WeaklyIncreasing}
in the loop invariant of the implementation.

\input{Listings/is_sorted.c.tex}

Since our implementation uses \WeaklyIncreasing in its loop invariant, and
follows the same principle in its code, its verification is
straight-forward---except for the final reasoning that
\inl{WeaklyIncreasing(a,n)} implies \inl{Increasing(a,n)}.

We have the lemma \logicref{WeaklyIncreasingIncreasing} for that step,
which needs to be proven manually with \coq.
%
The converse lemma \logicref{IncreasingWeaklyIncreasing}
is proven automatically, but isn't actually needed to verify our
\issorted implementation.
Alternatively, we could have dragged the predicate \Increasing along the
loop, which happens to cause no particular problems in this case.

\clearpage

