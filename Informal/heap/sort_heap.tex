\section{The \sortheap algorithm}
\Label{sec:sortheap}

Whereas in the \cxx Standard Library \cite[\S 28.7.7.4]{cxx-17-draft}
\sortheap works on a range of random access iterators,
our version operates on an array of \valuetype.
We therefore use the following signature for \sortheap

\begin{lstlisting}[style = acsl-block]

    void sort_heap(value_type* a, size_type n);
\end{lstlisting}

The function \sortheap rearranges the elements of a given heap \inl{a[0..n-1]}
in increasing order.
Thus, applying \sortheap to the heap in Figure~\ref{fig:heap-array}
produces the increasing array in Figure~\ref{fig:sortheap-array-post}.

\begin{figure}[hbt]
\centering
\includegraphics[width=0.65\linewidth]{Figures/sort_heap_array_post.pdf}
\caption{\Label{fig:sortheap-array-post}Array after the call of \sortheap}
\end{figure}

\FloatBarrier

\subsection{Formal specification of \sortheap}

The following listing shows our specification of \sortheap.
The formal specification of \sortheap must ensure that the
resulting array is increasing.
Furthermore the multiset contained by the array must be the same
as in the pre-state of the function.
The postconditions \inl{increasing} and \inl{reorder} express these properties, respectively.
The specification effort is relatively simple because we can reuse

\input{Listings/sort_heap.h.tex}

\clearpage

\subsection{Implementation of \sortheap}

The implementation of \sortheap is relatively simple because it relies on
\specref{popheap} performing essential work.
Our implementation of \sortheap repeatedly calls \popheap
to extract the maximum of the shrinking heap and adding it
to the part of the array that is already in increasing order.
The loop invariants of \sortheap describe the content of the array
in two parts.
The first \inl{i} elements form a heap and are described by the \inl{heap}
invariant.
The last \inl{n-i} elements are already arranged in increasing order.

As already mentioned in the introduction of Chapter~\ref{cha:binary-search},
we use the predicate \logicref{WeaklyIncreasing} for the loop annotation \inl{increasing}.
Thus, after leaving the loop we have in fact ``only'' shown that \inl{WeaklyIncreasing(a, n)}
holds.
In order to derive from this fact the final assertion \inl{increasing} that uses
the predicate \logicref{Increasing} we rely on lemma \logicref{WeaklyIncreasingIncreasing}.

\input{Listings/sort_heap.c.tex}

To verify the property \inl{reorder} we rely on the lemmas \logicref{MultisetReorder}
that express that the properties 

\begin{itemize}
\item \inl{MultisetReorder\{K,L\}(a, 0, i)} and
\item \inl{Unchanged\{Old,Here\}(a, i, n)}
\end{itemize}

imply the desired loop invariant \inl{MultisetReorder\{K,L\}(a, 0, n)}.

%\clearpage

