
\section{Basic heap concepts}
\Label{sec:heap-concepts}

The description of heaps at the beginning of this chapter is of course fairly vague.
It outlines only the most important properties
of various operations but does not clearly state what specific and verifiable
properties a range must satisfy such that it may be called a heap.

A more detailed description can be found in the Apache \cxx Standard Library User's Guide:\footnote{
  See \url{http://stdcxx.apache.org/doc/stdlibug/14-7.html}
}

\begin{quote}
A heap is a binary tree in which every node is larger than the values
associated with either child. A heap and a binary tree, for that matter,
can be very efficiently stored in a vector, by placing the children of
node $i$
at positions $2i + 1$ and $2i + 2$.
\end{quote}

We have, in other words, the following basic relations between indices of a heap:

\begin{align}
\Label{eq:heap-left}
   &\text{left child for index $i$}   && \mathrm{child_l}: i \mapsto 2i + 1  \\
\Label{eq:heap-right}
   &\text{right child for index $i$}  && \mathrm{child_r}: i \mapsto 2i + 2  \\
\intertext{and}
\Label{eq:heap-parent}
   &\text{parent index for index $i$}  && \mathrm{parent}: i \mapsto \frac{i - 1}{2}
\end{align}

%\clearpage 

These function are related through the following two equations
that hold for all integers~$i$.
Note that in \acsl integer division rounds towards zero (cf.\ \cite[\S 2.2.4]{ACSLSpec}).

\begin{align}
\Label{eq:heap-parent-left}
   \mathrm{parent}(\mathrm{child_l}(i)) &= i \\
\Label{eq:heap-parent-right}
   \mathrm{parent}(\mathrm{child_r}(i)) &= i
\end{align}


In order to given an example for the usefulness of heaps
we consider the following multiset of integers $X$.

\begin{align}
\Label{eq:heap-multiset}
  X &= \{2,3,3,3,6,7,8,8,9,11,13,14\}
\end{align}

\clearpage

Figure~\ref{fig:heap-tree} shows how the multiset from Equation~\eqref{eq:heap-multiset} 
can, according to the parent-child relations of a heap, be represented as a tree.

\begin{figure}[hbt]
\centering
\includegraphics[width=0.75\linewidth]{Figures/heap_tree_color.pdf}
\caption{\Label{fig:heap-tree}Tree representation of the multiset~$X$}
\end{figure}

\FloatBarrier

The numbers outside the nodes in Figure~\ref{fig:heap-tree} are the indices at which
the respective node value is stored in the underlying array of a heap (cf.\ Figure~\ref{fig:heap-array}).

\begin{figure}[hbt]
\centering
\includegraphics[width=0.65\linewidth]{Figures/heap_array_color.pdf}
\caption{\Label{fig:heap-array}Underlying array of a heap}
\end{figure}

\FloatBarrier
\clearpage

It is important to understand that there can be various representations of a multiset
as a heap.
Figure~\ref{fig:heap-alternative-tree}, for example, arranges the elements of
the multiset~$X$ as a heap in a different tree.

\begin{figure}[hbt]
\centering
\includegraphics[width=0.75\linewidth]{Figures/heap_tree_alternative_color.pdf}
\caption{\Label{fig:heap-alternative-tree} An alternative representation of the multiset~$X$}
\end{figure}

\FloatBarrier

Figure~\ref{fig:heap-array-alternative} then shows the underlying array that 
corresponds to the tree in Figure~\ref{fig:heap-alternative-tree}.

\begin{figure}[hbt]
\centering
\includegraphics[width=0.65\linewidth]{Figures/heap_array_alternative_color.pdf}
\caption{\Label{fig:heap-array-alternative}Underlying array of the alternative representation}
\end{figure}

\FloatBarrier
\clearpage

