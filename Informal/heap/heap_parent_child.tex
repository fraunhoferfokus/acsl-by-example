
\section{The auxiliary functions \heapparent and \heapchild}
\label{sec:auxiliary-heap-functions}
\label{sec:heapparent}
\label{sec:heapchild}

This section features the two auxiliary heap functions
We start with the function \specref{heapparent}
which is in principle the \isoc~counterpart of the \acsl function \logicref{HeapParent}.
We say \emph{in principle} because our definition avoids
the border case of the parent node of~0.

\input{Listings/heap_parent.h.tex}

Neither do we provide exact \isoc-counterparts for
the logic functions \logicref{HeapLeft} and \logicref{HeapRight}.
In fact, we have encountered only one situation (in the implementation of \implref{popheap}),
where such functions would have been useful.
However, what we really need in \popheap is to determine
for a given index~\inl{p} a child index~\inl{c} where the maximum
of the respective values \inl{a[HeapLeft(p)]} and\\
\inl{a[HeapRight(p)]} resides.
This computation is performed by the function \specref{heapchild}.

\input{Listings/heap_child.h.tex}

\clearpage

Note that in the implementation of \implref{heapchild}
we explicitly handle the case that the computation of child indices
could overflow. If this occurs, the function \heapchild returns~\inl{n}.

\input{Listings/heap_child.c.tex}


