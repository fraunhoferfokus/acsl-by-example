
\section{Representation of heap concepts in \acsl}
\Label{sec:heap-acsl}

The following listing shows three logic functions
\HeapLeft, \HeapRight and \HeapParent
that correspond to the definitions~\eqref{eq:heap-left},
\eqref{eq:heap-right} and~\eqref{eq:heap-parent}, respectively.
This listing also contains a number of \acsl lemma that state among other things that

\begin{itemize}
\item
the \HeapParent function satisfies the equations~\eqref{eq:heap-parent-left}
and~\eqref{eq:heap-parent-right} and
\item
the function \HeapParent 
is the \emph{left inverse} to the \HeapLeft and \HeapRight functions.\footnote{
 See Section \emph{Left and right inverses} at
 \url{http://en.wikipedia.org/wiki/Inverse_function}
}
\end{itemize}

\input{Listings/HeapNodes.acsl.tex}

\clearpage

On top of these basic definitions we introduce the predicate \logicref{Heap}.
The fact that element at index~0 of a (maximum) heap, is always the largest element of the heap
is express by Lemma \logicref{HeapMaximum} using the predicate \logicref{MaxElement}.

\input{Listings/Heap.acsl.tex}

The lemmas \HeapShrink and \HeapUnchanged formulate simple rules to
``transfer'' the heap property from an array to a related (sub-)array.

The predicate \HeapCompatible expresses under which
conditions the changing of an individual heap element does maintain the heap
property.
This predicate together with lemma \HeapCompatibleUpdate 
will be useful in the verification of the 
algorithms \implref{pushheap} and \implref{popheap}.

\clearpage

