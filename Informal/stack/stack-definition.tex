
\section{The structure \stacktype and its associated functions}
\Label{sec:stack-definition}

We now introduce one possible \isoc implementation of the above axioms.
It is centred around the \isoc structure \stacktype shown in the following listing.

\begin{listing}[hbt]
\centering
\begin{minipage}{0.9\textwidth}
\lstinputlisting[style=acsl-block, frame=single]{Source/stack.h}
\end{minipage}
\caption{\Label{lst:stack-definition}Definition of type \stacktype}
\end{listing}

This struct holds an array \inl{obj} of positive length called \inl{capacity}. 
The capacity of a stack is the maximum number of elements this stack can hold.
The field \inl{size} indicates the number elements that
are currently in the stack.
See also Figure~\ref{fig:stack-struct} which attempts to interpret
this definition according to Figure~\ref{fig:stack}.

\begin{figure}[hbt]
\centering
\includegraphics[width=0.60\linewidth]{Figures/stack-struct.pdf}
\caption{\Label{fig:stack-struct}Interpreting the data structure \stacktype}
\end{figure}

\FloatBarrier
\clearpage

Based on the stack functions from \S\ref{sec:stack-axioms},
we declare in the next listing the following functions as part of
our \stacktype data type.

\begin{listing}[hbt]
\centering
\begin{minipage}{0.9\textwidth}
\lstinputlisting[style=acsl-block, frame=single]{Source/stack_functions.h}
\end{minipage}
\caption{\Label{lst:stack-functions}Declaration of functions of type \stacktype}
\end{listing}

Most of these functions directly correspond to methods of the
\cxx \inl{std::stack} template class \cite[\S 26.6.6.1]{cxx-17-draft}.
The function \stackequal corresponds to the comparison operator~\inl{==},
whereas one use of \stackinit is to bring a stack into a
well-defined initial state.
The function \stackfull has no counterpart in \inl{std::stack}.
This reflects the fact that we 
avoid dynamic memory allocation, while \inl{std::stack} does not.

