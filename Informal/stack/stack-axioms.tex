
\section{Stack axioms}
\Label{sec:stack-axioms}

To specify the interplay of the stack access functions,
we use a set of axioms\footnote{
There is an analogy in geometry:
Euclid (e.g.\ \cite{Fitzpatrick.2008}) invented the use of
axioms there, but still kept definitions of \emph{point},
\emph{line}, \emph{plane}, etc.
Hilbert \cite{Hilbert.1968} recognized that the latter are not
only unformalizable, but also unnecessary, and dropped them,
keeping only the formal descriptions of relations between them.
},
all but one of them having the form of a conditional equation.


Let $V$ denote an arbitrary type.
We denote by $S_c$ the type of stacks with capacity $c > 0$ of
elements of type $V$.
The aforementioned functions then have the following signatures.

\begin{align*}
\mathrm{init} &:  S_c \rightarrow S_c, \\
\mathrm{push} &: S_c\times V \rightarrow S_c, \\
\mathrm{pop} &: S_c \rightarrow S_c,\\
\mathrm{top} &: S_c \rightarrow V, \\
\mathrm{size} &: S_c \rightarrow \mathbb{N}.\\
\end{align*}

With $\mathbb{B}$ denoting the \emph{boolean}
type we will also define two auxiliary functions
\begin{align*}
\mathrm{empty} &: S_c \rightarrow \mathbb{B},\\
\mathrm{full} &: S_c \rightarrow \mathbb{B}.
\end{align*}

To qualify as a stack these functions must satisfy the following rules
which are also referred to as \emph{stack axioms}.

\subsection{Stack initialization}

After a stack has been initialized its size is~0.
\begin{align}
\Label{eq:stack-init-size}
\mathrm{size}(\mathrm{init}(s)) &= 0.
\end{align}

The auxiliary functions $\mathrm{empty}$ and $\mathrm{full}$
are defined as follows
\begin{align}
\Label{eq:empty-stack}
\mathrm{empty}(s), & \qquad\text{iff}\qquad \mathrm{size(s)} = 0,  \\
\Label{eq:full-stack}
\mathrm{full}(s), & \qquad\text{iff}\qquad  \mathrm{size(s)} = c.
\end{align}

We expect that for every stack $s$ the following condition holds
\begin{align}
\Label{eq:stack-invariant}
 0 \leq \mathrm{size}(s) \leq c.
\end{align}

\clearpage

\subsection{Adding an element to a stack}

To push an element $v$ on a stack the stack must not be full.
If an element has been pushed on an eligible stack, its size increases by~1
\begin{align}
\Label{eq:stack-size-push}
\mathrm{size}(\mathrm{push}(s, v)) &= \mathrm{size}(s)+1, 
  &\text{if}\quad  \neg\mathrm{full}(s).\\
\intertext{Moreover, the element pushed on a stack is the top element of the resulting stack}
\Label{eq:stack-top-push}
\mathrm{top}(\mathrm{push}(s, v)) &= v, 
  &\text{if}\quad \neg\mathrm{full}(s).
\end{align}

\subsection{Removing an element from a stack}

An element can only be removed from a non-empty stack.
If an element has been removed from an eligible stack the
stack size decreases by~1
\begin{align}
\Label{eq:stack-size-pop}
\mathrm{size}(\mathrm{pop}(s)) &= \mathrm{size}(s)-1,
  &&\text{if}\quad \neg\mathrm{empty}(s).
%
\intertext{
If an element is pushed on a stack and immediately afterwards
an element is removed from the resulting stack then the final stack
is equal to the original stack}
%
\Label{eq:stack-pop-push}
\mathrm{pop}(\mathrm{push}(s, v)) &= s,
  &&\text{if}\quad \neg\mathrm{full}(s). \\
\intertext{Conversely, if an element is removed from a non-empty stack
  and if afterwards the top element of the original stack is
  pushed on the new stack
  then the resulting stack is equal to the original stack.}
%
\Label{eq:stack-push-pop-top}
\mathrm{push}(\mathrm{pop}(s),\mathrm{top}(s)) &= s, &&\text{if}\quad \neg\mathrm{empty}(s). 
\end{align}

\clearpage

\subsection{A note on exception handling}

We don't impose a requirement on \inl{push(s, v)} if \inl{s}
is a full stack, nor on \inl{pop(s)} or \inl{top(s)} if \inl{s} is an
empty stack.
%
Specifying the behavior in such \emph{exceptional} situations is a
problem by its own; a variety of approaches is discussed in the
literature.
%
We won't elaborate further on this issue, but only give an example to warn
about ``innocent-looking'' exception specifications that may lead to
undesired results.

If we'd introduce an additional error value \inl{err} in the element type 
$V$ and require \inl{top(s) = err} if \inl{s} is empty, we'd be faced
with the problem of specifying the behavior of \inl{push(s, err)}.
%
At first glance, it would seem a good idea to have \inl{err} just been
ignored by \inl{push}, i.e.\ to require
\begin{align}
\Label{eq:err}
\mathrm{push}(s,\mathrm{err}) & = s.
\end{align}

However, we then could derive for any non-full and non-empty stack \inl{s}, that

\begin{align*}
\mathrm{size}(s)
   &=  \mathrm{size}(\mathrm{pop}(\mathrm{push}(s, \mathrm{err}))) && \text{by \ref{eq:stack-pop-push}}\\
   &=  \mathrm{size}(\mathrm{pop}(s)) && \text{as assumed in \ref{eq:err}}\\
   &=  \mathrm{size}(s) - 1 && \text{by \ref{eq:stack-size-pop}}\\
\end{align*}
%
i.e.\ no such stacks could exist, or all \inl{int} values would be equal.


