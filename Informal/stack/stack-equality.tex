
\section{Equality of stacks}
\Label{sec:stack-equality}

Defining equality of instances of non-trivial data types, in
particular in object-oriented languages, is not an easy task.
%
The book \emph{Programming in Scala}\cite[Chapter~28]{OderskyEtAl2008} 
devotes to this topic a whole chapter of more than twenty pages.
%
In the following two sections we give a few hints how \acsl
and \framac can help to
correctly define equality for a simple data type.

We consider two stacks as equal if they have the same size and if they contain the same objects.
%
To be more precise, let~\inl{s} and~\inl{t} two pointers of type \stacktype,
then we define the predicate \StackEqual as in the following listing.

\input{Listings/Stack.acsl.tex}

Our use of labels in this listing makes
the specification somewhat hard to read (in particular in the last line
where we reuse the predicate \logicref{Equal}.
%
However, this definition of \StackEqual will allow us later to compare 
the same stack object at different points of a program.
%
The logical expression \inl{StackEqual\{A,B\}(s,t)}
reads informally as: 
{The stack object \inl{*s} at program point \inl{A}
equals the stack object \inl{*t} at program point \inl{B}}.

The reader might wonder why we exclude the capacity of a stack
into the definition of stack equality.
This approach can be motivated with the behavior of the method
\inl{capacity} of the class \inl{std::vector<T>}.
There, equal instances of type \inl{std::vector<T>} may very well 
have different capacities.\footnote{
See \url{http://www.cplusplus.com/reference/vector/vector/capacity}
}

If equal stacks can have different capacities then, according to our
definition of the predicate \logicref{StackFull}, 
we can have to equal stacks where one is full and the other one is not.

A finer, but very important point in our specification of equality
of stacks is that the elements of the arrays \inl{s->obj} and \inl{t->obj}
are compared only up to \inl{s->size} and \emph{not} up  to
\inl{s->capacity}.
Thus the two stacks \inl{s} and \inl{t} in Figure~\ref{fig:equal-stacks}
are considered
equal although there is are obvious differences in their internal arrays.

\begin{figure}[hbt]
\centering
\includegraphics[width=0.75\linewidth]{Figures/stack12.pdf}
\caption{\Label{fig:equal-stacks} Example of two equal stacks}
\end{figure}

\FloatBarrier

If we define an equality relation $(=)$ of objects for a data
type such as \stacktype,
we have to make sure that the following rules hold.

\begin{subequations}
\Label{eq:equivalence-relation}
\begin{align}
   \text{reflexivity}\qquad && \forall s \in S&: s = s,\\
   \text{symmetry}\qquad &&    \forall s,t \in S&: s = t \implies t = s,\\
   \text{transitivity}\qquad &&    \forall s,t,u \in S&: s = t \land t = u \implies s = u.
\end{align}
\end{subequations}


Any relation that satisfies the conditions~\eqref{eq:equivalence-relation}
is referred to as an \emph{equivalence relation}.
%
The mathematical set of all instances that are considered equal to
some given instance \inl{s} is called the equivalence class of \inl{s}
with respect to that relation.

Our formalization of \logicref{StackEquality} shows 
these three rules for the relation \StackEqual;
it can be automatically verified that they are a consequence of the
definition of \StackEqual.

The two stacks in Figure~\ref{fig:equal-stacks} show that
an equivalence class of \StackEqual
can contain more than one element.\footnote{
    This is a common situation in mathematics. For example,
    the equivalence class of
    the rational number $\frac{1}{2}$ contains infinitely many elements,
    viz.\ $\frac{1}{2},
    \frac{2}{4}, \frac{7}{14}, \ldots$.
}
The stacks \inl{s} and \inl{t} in Figure~\ref{fig:equal-stacks}
are also referred to as two \emph{representatives} of the
same equivalence class.
In such a situation, the question arises whether a function
that is defined on
a set with an equivalence relation can be defined in such a
way that its definition
is \emph{independent of the chosen representatives}.\footnote{
    This is why mathematicians know that
    $\frac{1}{2} + \frac{3}{5}$
    equals $\frac{7}{14} + \frac{3}{5}$.
}
We ask, in other words, whether the function is \emph{well-defined}
on the set of all equivalence classes of the relation \StackEqual.\footnote{
  See \url{http://en.wikipedia.org/wiki/Well-definition}.
}
The question of well-definition
will play an important role when verifying the functions of
the \stacktype (see \S\ref{sec:stack-functions}).

