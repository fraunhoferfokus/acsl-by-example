
\section{Verification of stack axioms}
\Label{sec:stack-verification}

In this section we show that the stack functions defined in
\S\ref{sec:stack-functions}
satisfy the stack Axioms of \S\ref{sec:stack-axioms}.

The annotated code has been obtained from the axioms in a fully systematical way.
In order to transform a condition equation $p \rightarrow s = t$:
\begin{itemize}
\item Generate a clause \inl{requires p}.
\item Generate a clause \inl{requires x1 == ... == xn} 
for each variable \inl{x} with $n$ occurrences 
in $s$ and $t$.
\item Change the $i$-th occurrence of \inl{x} to \inl{xi} 
in $s$ and $t$.
\item Translate both terms $s$ and $t$ to reversed polish notation.
\item Generate a clause \inl{ensures y1 == y2}, where \inl{y1} and
\inl{y2} denote the value corresponding to 
the translated $s$ and $t$, respectively.
\end{itemize}

This makes it easy to implement a tool that does the translation automatically,
but yields a slightly longer contract in our example.


\subsection{Resetting a stack}
\Label{sec:axiomsizeofinit}


Our formulation in \acsl\slash\isoc  of the axiom in
Equation~\eqref{eq:stack-init-size} is shown in  the following
listing.

\input{Listings/axiom_size_of_init.c.tex}

\clearpage

\subsection{Adding an element to a stack}
\Label{sec:axiomsizeofpush}
\Label{sec:axiomtopofpush}

Axioms~\eqref{eq:stack-size-push} and~\eqref{eq:stack-top-push}
describe the behavior of a stack when an element is added.

\input{Listings/axiom_size_of_push.c.tex}

Except for the \inl{assigns} clauses, the \acsl specification refers
only to encapsulating logic functions and predicates defined in
\S\ref{sec:stack-invariants}.
If \acsl would provide a means to define encapsulating logic
functions returning also sets of memory locations,
the expressions in \inl{assigns} clauses would not need to refer to
the details of our
\stacktype implementation.\footnote{
In \cite[\S 2.3.4]{ACSLSpec}, a powerful sublanguage to
build memory location set expressions is defined. 
We will explore its capabilities in a later version.
}
As an alternative, \inl{assigns} clauses could be omitted, as long as
the proofs are only used to convince a human reader.

\input{Listings/axiom_top_of_push.c.tex}

\clearpage

\subsection{Removing an element from a stack}
\Label{sec:axiompopofpush}
\Label{sec:axiomsizeofpop}
\Label{sec:axiompushofpoptop}

This section shows the Listings for Axioms~\ref{eq:stack-size-pop}, \ref{eq:stack-pop-push}
and~\ref{eq:stack-push-pop-top} which
describe the behavior of a stack when an element is removed.

\input{Listings/axiom_size_of_pop.c.tex}
\input{Listings/axiom_pop_of_push.c.tex}
\input{Listings/axiom_push_of_pop_top.c.tex}

