
\subsection{The function \stackpush}
\Label{sec:stackpush}
\Label{sec:stackpushwd}

The following listing shows the specification of the function \stackpush.
In accordance with Axiom~\eqref{eq:stack-size-push}, \stackpush is supposed
to increase the number of elements of a non-full stack by one.
The specification also demands that the value that is pushed on a
non-full stack becomes the top element of the resulting stack (see 
Axiom~\eqref{eq:stack-top-push}).

\input{Listings/stack_push.h.tex}

The implementation of \stackpush is shown in the next listing.
It checks whether its argument is a non-full stack in which case it
increases the field \inl{size} by one but only after it has assigned
the function argument to the element \inl{obj[size]}.

\input{Listings/stack_push.c.tex}

\vfill

The following listing shows our formalization of the well-definition for \stackpush.
%
The function \stackpush does not return a value but rather modifies its argument.
For the well-definition of \stackpush we therefore check whether it
turns equal stacks into equal stacks.

\input{Listings/stack_push_wd.c.tex}

However, equality of the stack arguments is not sufficient for a proof
that \stackpush is well-defined.
We must also ensure that there is no \emph{aliasing} between the two stacks.
Otherwise modifying one stack could modify the other stack in unexpected ways.
In order to express that there is no aliasing between two stacks, 
we use the predicate \logicref{StackSeparated}.

In order to achieve an automatic verification of \implref{stackpushwd} we
have added the assertions \inl{top} and \equal and introduced the lemma
\logicref{StackPushEqual} in the following listing.

\input{Listings/StackLemmas.acsl.tex}

