
\section{Verification of stack functions}
\Label{sec:stack-functions}

In this section we verify the functions
\begin{itemize}
\item \stackequal (\S\ref{sec:stackequal})
\item \stackinit (\S\ref{sec:stackinit})
\item \stacksize (\S\ref{sec:stacksize})
\item \stackfull (\S\ref{sec:stackfull})
\item \stackempty (\S\ref{sec:stackempty}) 
\item \stacktop (\S\ref{sec:stacktop})
\item \stackpush (\S\ref{sec:stackpush})
\item \stackpop (\S\ref{sec:stackpop})
\end{itemize}

of the data type \stacktype.
To be more precise, we provide for each of function \inl{stack_foo}:
\begin{itemize}
\item an \acsl specification of \inl{stack_foo} 
\item a \isoc implementation of \inl{stack_foo}
\item a \isoc function \inl{stack_foo_wd}\footnote{
  The suffix \inl{_wd} stands for \emph{well definition}
}
accompanied by a an \acsl contract that expresses that
the implementation of \inl{stack_foo} is well-defined.
Figure~\ref{fig:methodology-wd}
shows our methodology for the verification of
well-definition in the \inl{pop} example,
$(\approx)$ again indicating the user-defined \stacktype equality.
\end{itemize}

\begin{figure}[hbt]
\centering
\includegraphics[width=0.95\linewidth]{Figures/stack_pop_wd.pdf}
\caption{Methodology for the verification of well-definition}
\Label{fig:methodology-wd}
\end{figure}

\FloatBarrier

Note that the specifications of the various functions will explicitly
refer to the \emph{internal state} of \stacktype.
In \S\ref{sec:stack-verification} we will show that the
\emph{interplay} of these functions satisfy the stack axioms from
\S\ref{sec:stack-axioms}.


\clearpage

\stackequal & \ref{sec:stack-runtime-equality} & 22   &  22   & 100  &   7    &  15    &   0    &   0 \\\hline\clearpage

\subsection{The function \stackinit}
\Label{sec:stackinit}

The following listing shows the specification of \stackinit.
Note that our specification of the post-conditions contains a redundancy
because a stack is empty if and only if its size is zero.

\input{Listings/stack_init.h.tex}

The next listing shows the implementation of \stackinit.
It simply initializes \inl{obj} and \inl{capacity} with the respective
value of the array and sets the field \inl{size} to zero.

\input{Listings/stack_init.c.tex}

\clearpage
\stacksize & \ref{sec:stacksize} & 6    &   6   & 100  &   1    &   5    &   0    &   0 \\\hline\clearpage
\stackfull & \ref{sec:stackfull} & 13   &  13   & 100  &   5  &   8  &   0  &   0  &   0 \\\hline
\clearpage
\stackempty & \ref{sec:stackempty} & 12   &  12   & 100  &   5  &   0  &   7  &   0  &   0 \\\hline
\clearpage
\stacktop & \ref{sec:stacktop} & 18   &  18   & 100  &   6  &   0  &  12  &   0  &   0 \\\hline


\subsection{The function \stackpush}
\Label{sec:stackpush}
\Label{sec:stackpushwd}

The following listing shows the specification of the function \stackpush.
In accordance with Axiom~\eqref{eq:stack-size-push}, \stackpush is supposed
to increase the number of elements of a non-full stack by one.
The specification also demands that the value that is pushed on a
non-full stack becomes the top element of the resulting stack (see 
Axiom~\eqref{eq:stack-top-push}).

\input{Listings/stack_push.h.tex}

The implementation of \stackpush is shown in the next listing.
It checks whether its argument is a non-full stack in which case it
increases the field \inl{size} by one but only after it has assigned
the function argument to the element \inl{obj[size]}.

\input{Listings/stack_push.c.tex}

\vfill

The following listing shows our formalization of the well-definition for \stackpush.
%
The function \stackpush does not return a value but rather modifies its argument.
For the well-definition of \stackpush we therefore check whether it
turns equal stacks into equal stacks.

\input{Listings/stack_push_wd.c.tex}

However, equality of the stack arguments is not sufficient for a proof
that \stackpush is well-defined.
We must also ensure that there is no \emph{aliasing} between the two stacks.
Otherwise modifying one stack could modify the other stack in unexpected ways.
In order to express that there is no aliasing between two stacks, 
we use the predicate \logicref{StackSeparated}.

In order to achieve an automatic verification of \implref{stackpushwd} we
have added the assertions \inl{top} and \equal and introduced the lemma
\logicref{StackPushEqual} in the following listing.

\input{Listings/StackLemmas.acsl.tex}

\clearpage
\stackpop & \ref{sec:stackpop} & 34   &  34   & 100  &  20  &  14  &   0  &   0  &   0 \\\hline
\clearpage
\clearpage

