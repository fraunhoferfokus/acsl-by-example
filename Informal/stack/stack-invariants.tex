
\section{Stack invariants}
\Label{sec:stack-invariants}

Not every possible instance of type \stacktype is considered a
valid one, e.g., with our definition of \stacktype in Listing~\ref{lst:stack-definition},
\inl{Stack s = \{\{0,0,0,0\},4,5\}} is not.
In the following listing, we present
basic logic functions and predicates that we will use
throughout this chapter
In particular, we  define the predicate \logicref{StackInvariant} that
discriminates valid and invalid instances.

\input{Listings/StackInvariant.acsl.tex}

We start, with the auxiliary logic function
\StackCapacity, \StackSize and \StackStorage
which we can use in specifications to refer
to the fields \inl{capacity}, \inl{size} and \inl{obj} of \stacktype, respectively.
%
This listing also contains the logic function \StackTop which defines
the array element with index \inl{size - 1} as the top place of a stack.

The reader can consider this as an attempt to hide
implementation details from
the specification.
%
We intentionally use here integer as a return value of these logic functions.
Inaccurate use of logic functions with bounded types in axioms with
arithmetic operations may lead to inconsistencies.

We also introduce the predicates \logicref{StackEmpty} and \logicref{StackFull}
that express the concepts of empty and full stacks
by referring to a stack's size and capacity (see Equations~\eqref{eq:empty-stack}
and~\eqref{eq:full-stack}).

There are some obvious invariants that must be fulfilled by every
valid object of type \stacktype:
\begin{itemize}
\item The stack capacity shall be strictly greater than zero
      (an empty stack is ok but a stack that cannot hold anything is not useful).

\item The pointer \inl{obj} shall refer to an array of length \inl{capacity}.

\item The number of elements \inl{size} of a stack the must
      be non-negative and not greater than its capacity.
\end{itemize}

These invariants are all formalized in the predicate \logicref{StackInvariant}.

Note how the use of the previously defined logic functions and predicates
allows us to define the stack invariant without
directly referring to the fields of \stacktype.

We sometimes wish to express that there is no \emph{memory aliasing} between two stacks.
If there were aliasing, then modifying one stack could modify the other
stack in unexpected ways.
In order to express that there is no aliasing between two stacks,
we define the predicate \StackSeparated in the next listing.

\input{Listings/StackUtility.acsl.tex}

This listing also contains the predicate \logicref{StackUnchanged}
that we will use to describe cases that the contents of a stack hasn't changed.

