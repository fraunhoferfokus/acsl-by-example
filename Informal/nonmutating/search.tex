
\section{The \search algorithm}
\Label{sec:search}

The \search algorithm in the \cxx Standard Library \cite[\S
28.5.13]{cxx-17-draft} finds a
subsequence that is identical to a given sequence when 
compared element-by-element.
For our purposes we have modified
the generic implementation
to that of an array of type \valuetype.
The signature now reads:

\begin{lstlisting}[style = acsl-block]

    size_type search(const value_type* a, size_type n,
                     const value_type* b, size_type p);
\end{lstlisting}

The function \search returns the first
index \inl{s} of the array \inl{a} where the condition \inl{a[s+k] == b[k]} holds for each
index~\inl{k} with \inl{0 <= k < p}
(see Figure~\ref{fig:search}).
If no such index exists, then \search returns the length
\inl{n} of the array \inl{a}.

\begin{figure}[hbt]
\centering
\includegraphics[width=0.69\textwidth]{Figures/search.pdf}
\caption{\Label{fig:search} Searching the first occurrence of \inl{b[0..p-1]} in \inl{a[0..n-1]}}
\end{figure}


\subsection{The predicate \HasSubRange}

Our specification of \search starts with introducing the
predicate~\logicref{HasSubRange}.
This predicate formalizes, using the predicate \logicref{Equal},
that the sequence~\inl{a} contains a subsequence which equal the sequence~\inl{b}.
Of course, in order to contain a subsequence of length \inl{p},
\inl{a} must be at least that large; this is expressed by lemma 
\HasSubRangeSizes.

\input{Listings/HasSubRange.acsl.tex}

\subsection{Formal specification of \search}

The following listing shows the specification of \specref{search}.

\input{Listings/search.h.tex}

Conceptually, the specification of \search is very similar to that of
\specref{find}.
We therefore use again two behaviors to capture the essential aspects of \search.

\begin{itemize}
\item
The behavior \inl{has_match} applies if the sequence \inl{a} 
contains a subsequence identical to \inl{b}. 
We express this condition with \inl{assumes} using the predicate 
\logicref{HasSubRange}.

The ensures clause \inl{bound} of behavior \inl{has_match} 
indicates that the returned index value must be in the range~\inl{[0..n-p]}.
The clause \inl{result} expresses that
\search returns an index where a copy of \inl{b} can be found in \inl{a}.
Clause \inl{first}
indicates that the least index with that property is returned,
i.e.\ that \inl{b} can't be found in \inl{a[0..\\result+p-2]}.

\item
The behavior \inl{no_match} covers the case that there is no
subsequence \inl{a} that equals
\inl{b}. In this case, \search must return the length \inl{n} of the
range \inl{a}.
%
If the ranges \inl{a} or \inl{b} are empty
then the return value will be \inl{0}. 
\end{itemize}

The formula in the assumes clause of the behavior
\inl{has_match} is the negation of the assumes clause of the 
behavior \inl{no_match}.
Therefore, we can express that these two behaviors are
\emph{complete} and \emph{disjoint}.

\clearpage

\subsection{Implementation of \search}
\Label{subsec:search implementation}

The implementation of \implref{search} is relatively easy to understand, but
needs an order of magnitude of \inl{n*p} operations.
In contrast, the sophisticated algorithm from
\cite{Knuth.Morris.Pratt.1977}
needs only~\inl{n+p} operations.\footnote{
  \Label{fn:search implementation efficiency}
  The efficiency question has been also discussed by the \cxx standardization committee, 
  see \url{http://www.open-std.org/jtc1/sc22/wg21/docs/papers/2014/n3905.html}
}

The loop invariant \inl{not_found} is needed for the proof of the
postconditions of the behavior \inl{has_match} in the contract of \specref{search}.
It expresses that the subsequence \inl{b} has not been found up to the current iteration step.
%
Neither \inl{p == 0} nor \inl{n == 0} need to be handled separately, not
even for efficiency reasons:
in the former case, \inl{equal(a+i, p, b)} will succeed in the first iteration,
while in the latter, \inl{p > n} will apply.

\input{Listings/search.c.tex}

\clearpage

