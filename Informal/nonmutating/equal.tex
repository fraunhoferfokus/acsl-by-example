
\section{The \equal and \mismatch algorithms}
\Label{sec:equal}
\Label{sec:mismatch}

The algorithms \equal \cite[\S 28.5.11]{cxx-17-draft}
and \mismatch \cite[\S 28.5.10]{cxx-17-draft}
of the \cxx Standard Library compare two generic sequences.
For our purposes we have modified the generic implementation
to that of an array of type \valuetype.
The signatures read

\begin{lstlisting}[style = acsl-block]

  bool       equal(const value_type* a, size_type n, const value_type* b);

  size_type  mismatch(const value_type* a, size_type n, const value_type* b);
\end{lstlisting}

The function \equal returns~\inl{true} if and only if
\inl{a[i] == b[i]} holds for each \inl{0 <= i < n}.
Otherwise,~\equal returns~\inl{false}.

The \mismatch algorithm is slightly more general than the negation of \equal: 
it returns the smallest index where the two ranges~\inl{a} and~\inl{b} differ.
If no such index exists, that is, if both ranges are equal, then \mismatch
returns the (common) length~\inl{n} of the two ranges.

\subsection{The \Equal predicate}

The fact that two arrays \inl{a[0]..a[n-1]}
and \inl{b[0]..b[n-1]} are equal when compared element by element,
is a property we might need again in other specifications, as it
describes a very basic property.

The motto \emph{don't repeat yourself} is not just good programming
practice.\footnote{
    Compare \url{http://en.wikipedia.org/wiki/Don't_repeat_yourself}
}
It is also true for concise and easy to understand specifications.
We will therefore introduce specification elements that we can apply to
the \equal algorithm as well as to other specifications and
implementations with the described property.

We start with introducing several \emph{overloaded} versions
of the predicate~\logicref{Equal}.

\input{Listings/Equal.acsl.tex}

The letters \inl{K} and \inl{L} in the definition of \Equal
are so-called  \emph{labels}\footnote{
    Labels are used in \isoc to name the target of the \emph{goto}
    jump statement.
}
that refer to program states in which the ranges~\inl{a[..]} and~\inl{b[..]} are evaluated.
\framac defines several standard labels, e.g.\xspace \inl{Old}
and \inl{Post},
a programmer can use to refer to the pre-state or post-state,
respectively, of a function.
For more details on labels we refer to the \acsl
specification \cite[\S 2.6.9]{ACSLSpec}.


\subsection{Formal specification of \equal and \mismatch}

Using predicate \logicref{Equal} we can formulate the specification of \specref{equal}
using the predefined label~\inl{Here}. 
When used in an \inl{ensures} clause, the label~\inl{Here} refers to the post-state of a function.
Note that the equivalence is needed in the ensures clause.
Putting an equality instead is not legal in ACSL,
because \Equal is a predicate, not a function.

\input{Listings/equal.h.tex}


The formal specification of \specref{mismatch}
is more complex than that of \specref{equal} because the return value of mismatch
provides more information than just reporting whether the two arrays are equal.

\input{Listings/mismatch.h.tex}

On the other hand, the specification is conceptually quite similar to that of \specref{findii}.
%
While \findii returns the smallest index~\inl{i} where \inl{a[i] == v} holds,
\mismatch finds the smallest index \inl{a[i] != b[i]}.
Note in particular the use of \Equal in the specification of \mismatch.
As in the specification of \findii
the completeness and disjointness of \mismatch's behaviors is quite obvious,
because the \inl{assumes} clauses of
\inl{all_equal} and \inl{some_not_equal} are negations of each other.

%\clearpage

\subsection{Implementation of \equal and \mismatch}

The implementation of \implref{equal} consists of a simple call of \mismatch.

\input{Listings/equal.c.tex}

The implementation of \implref{mismatch}
has been enriched with some loop annotations to support the deductive verification.

\input{Listings/mismatch.c.tex}

We use again the predicate \logicref{Equal}
in order to express that all indices~\inl{k} that are less than the
current index~\inl{i}
satisfy the condition \inl{a[k] == b[k]}.
This is necessary to prove that \mismatch indeed returns the smallest index
where the two ranges differ.

\clearpage

