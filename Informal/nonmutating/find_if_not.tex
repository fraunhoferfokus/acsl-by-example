

\section{The \findifnot algorithm }
\label{sec:findifnot}

Many algorithms in the \cxx standard library can be parameterized not only by the type
of sequence but also using so-called \emph{function objects}.
One example is the \findifnot algorithm that accepts a 
\emph{predicate function object}~$P$.
The algorithm then returns the first position~$i$ in the input sequence where
$P(i)$ does \emph{not} hold.

While function objects could be emulated in \isoc with \emph{pointers to functions},
we will not follow this road here.
The main reason is that function pointers are, so far, only supported momentarily by \framac.
Moreover, there is as of now no support for parameterized \acsl predicates.
For these reasons our implementation of \findifnot only returns the first position
in an array where a given value does \emph{not} occur.
The signature, thus, reads 

\begin{lstlisting}[style=acsl-block]

       size_type find_if_not(const value_type* a, size_type n, value_type v);
\end{lstlisting}

On the one hand, this is not a very exciting addition to our
collections of verified algorithms.
It gives us, however, an opportunity to introduce the predicates
\logicref{AllEqual}
and \logicref{SomeNotEqual}
and more importantly the logic function
\logicref{FindNotEqual}
that will later play
an essential role in the specification of the algorithm \removecopy,
or more precisely, its variant \specref{removecopyiii}.

\input{Listings/AllSomeNot.acsl.tex}

\clearpage

The predicate \AllEqual expresses that each member of the array section

\inl{a[m..n-1]} equals~\inl{v}.
We also introduce the predicate \SomeNotEqual which is the negation of \AllEqual.
Both predicates complement the predicates 
\logicref{SomeEqual} and 
\logicref{NoneEqual}.

There are two additional overloaded versions of \AllEqual.
The first version uses the value \inl{a[m]} as \inl{v}.
The second version is just a shortcut when the first index~\inl{m} equals~0.

\subsection{The logic function \FindNotEqual}

The definition of the overloaded logic function \FindNotEqual is shown in
Listings~\ref{logic:FindNotEqual-1} and~\ref{logic:FindNotEqual-2}.
This function is very similar to 
\logicref{Find} except that it
finds the first element in a sequence that \emph{differs} from a given value.
%
Note that in lemma \FindNotEqualUnchanged we are using the predicate \logicref{Unchanged}
that will be defined in a later chapter.

\begin{logic}[hbt]
\begin{minipage}{\textwidth}
\lstinputlisting[linerange={1-41}, style=acsl-block, frame=single]{Source/FindNotEqual.acsl}
\end{minipage}
\caption{\label{logic:FindNotEqual-1}
   The logic function \FindNotEqual (1)}
\input{Listings/FindNotEqual.acsl.labels.tex}
\input{Listings/FindNotEqual.acsl.index.tex}
\end{logic}

\FloatBarrier

\begin{logic}[hbt]
\begin{minipage}{\textwidth}
\lstinputlisting[linerange={42-100}, style=acsl-block, frame=single]{Source/FindNotEqual.acsl}
\end{minipage}
\caption{\Label{logic:FindNotEqual-2}The logic function \FindNotEqual (2)}
\end{logic}

\FloatBarrier

\clearpage

\subsection{Formal specification of \findifnot}

The contract of \specref{findifnot} is, unsurprisingly,
very similar to that of \specref{findiii}.
The only difference is that we replaced 
\logicref{Find} by 
\logicref{FindNotEqual}.

\input{Listings/find_if_not.h.tex}

\subsection{Implementation of \findifnot}

The implementation of \implref{findifnot}
also has a lot of similarities with of \implref{findiii}.
Here again we have replaced \Find by \FindNotEqual and, of course,
we check in the loop body that the value \inl{a[i]} \emph{differs} from the
given value~\inl{v}.

\input{Listings/find_if_not.c.tex}

\clearpage

