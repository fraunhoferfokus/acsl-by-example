
\section{The \searchn algorithm}
\Label{sec:searchn}

The \searchn algorithm in the \cxx Standard Library \cite[\S
28.5.13]{cxx-17-draft} finds
the first place where a given value starts to occur a given number of
times in a given sequence.
For our purposes we have modified
the generic implementation
to that of an array of type \valuetype.
The signature now reads:

\begin{lstlisting}[style = acsl-block]

   size_type
   search_n(const value_type* a, size_type n, size_type p, value_type v);
\end{lstlisting}

Note the similarity to the signature of \search (\S\ref{sec:search}).
The only difference is that \inl{v} now is a single value rather than
an array.

\begin{figure}[hbt]
\centering
\includegraphics[width=0.59\textwidth]{Figures/search_n.pdf}
\caption{\Label{fig:searchn} Searching the first occurrence a given constant sequence in \inl{a[0..n-1]}}
\end{figure}

\FloatBarrier

The function \searchn returns the first
index \inl{s} of the array \inl{a} where the condition \inl{a[s+k] == v}
holds for each index~\inl{k} with \inl{0 <= k < p} (see Figure~\ref{fig:searchn}).
If no such index exists, then \searchn returns the length
\inl{n} of the array \inl{a}.


\subsection{The predicate \HasConstantSubRange}

Our specification of \searchn starts with introducing the
predicate 
\logicref{HasConstantSubRange}.

\input{Listings/HasConstantSubRange.acsl.tex}

This predicate formalizes that the sequence~\inl{a} of length \inl{n}
contains a subsequence of \inl{p} times the value~\inl{v}.
It thereby reuses the predicate
\logicref{AllEqual}.

Similar to predicate 
\logicref{HasSubRange},
in order to contain \inl{p} repetitions, the size of the
array \inl{a[0..n-1]} must be at least that large;
this is what lemma 
\logicref{HasConstantSubRangeSizes} says.

\subsection{Formal specification of \searchn}

Like for \specref{search}, our specification of \specref{searchn}
is very similar to that of \specref{findii}.

\input{Listings/search_n.h.tex}

We again use two behaviors to capture the essential aspects of \searchn.

\begin{itemize}
\item
The behavior \inl{has_match} applies if the sequence \inl{a} 
contains an \inl{n}-fold repetition of \inl{b}. 
We express this condition with  \inl{assumes} by using the predicate
\logicref{HasConstantSubRange}.
The \inl{result} ensures clause of behavior \inl{has_match} indicates
that the return value must be in the range~\inl{[0..n-p]}.
The \inl{match} ensures clause
expresses that the return value of \searchn actually points to an index 
where \inl{b}
can be found \inl{p} or more times in \inl{a}. 
The \inl{first} ensures clause expresses that the minimal index with
this property is returned.

\item
The behavior \inl{no_match} covers the case that there is no matching
subsequence in sequence \inl{a}.
In this case, \searchn must return the length \inl{n} of the
range \inl{a}.
\end{itemize}

\input{Listings/search_n.c.tex}

\subsection{Implementation of \searchn}

Although the specification of \specref{searchn} strongly resembles that of
\specref{search}, their implementations differ significantly.
The implementation of \implref{searchn} has a time complexity of
$\mathcal{O}(n)$, whereas the implementation of
\implref{search} employs an easy, but
a non-optimal algorithm needing $\mathcal{O}(n \cdot p)$ time.

Our implementation maintains in the variable \inl{start} the beginning
of the most recent consecutive range of values~\inl{v}.
The loop invariant \inl{not_found} states that we didn't find an
\inl{p}-fold repetition of \inl{b} up to now; if we find one, we
terminate the loop, returning \inl{start}.
%
We handle the boundary cases \inl{n < p} and \inl{p == 0} in explicit else branches.
We found this easier when trying to ensure a verification by automatic provers.

\clearpage

