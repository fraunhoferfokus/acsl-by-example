
\section{The \counti algorithm}
\Label{sec:counti}

The \counti algorithm in the \cxx Standard Library \cite[\S
28.5.9]{cxx-17-draft} counts
the frequency of occurrences for a particular element in
a sequence.
For our purposes we have modified
the generic implementation
to that of arrays of type \valuetype.
The signature now reads:

\begin{lstlisting}[style = acsl-block]

  size_type
  count(const value_type* a, size_type n, value_type v);
\end{lstlisting}

Informally, the function returns the number of occurrences of
\inl{v} in the array \inl{a}.

\subsection{The logic function \Count}

When trying to specify \counti we are faced with the situation that
\acsl does not provide a definition of counting a value in an array.\footnote{
  This statement is not quite true because the \acsl documentation 
  lists \inl{numof} as one of several \emph{higher order logic constructions} \cite[\S 2.6.7]{ACSLSpec}.
  However, these \emph{extended quantifiers} are mentioned only as experimental features.
}
We therefore start with an axiomatic definition of \emph{logic function} \Count
that captures the basic intuitive features of counting on an array section.
The expression \inl{Count(a,m,n,v)} returns the number of
occurrences of \inl{v} in \inl{a[m],...,a[n-1]}.

The specification of \counti will then be fairly short because it employs
our \emph{logic function}
\Count whose (considerably) longer definition is given in the 
Listings~\ref{logic:Count-1} and~\ref{logic:Count-2}.\footnote{
 This definition of \Count is a generalization of
 the \emph{logic function} \inl{nb_occ} of the \acsl specification \cite{ACSLSpec}.
}


\begin{itemize}
\item
The \acsl keyword \inl{axiomatic} 
is used to structure the specification and gather the logic function \Count and related lemmas.
Note that the interval bounds \inl{m} and \inl{n} and the return value for \Count are of type \inl{integer}.

\item
The logic functions \Count is recursively defined.
It consist of two checks: whether the range is empty and for the value of
the "current" element in the array. The recursion goes down on the range length.
We also provide an overloaded version of \Count that accepts only
the length of an array, thus relieving the use the supply the argument $m=0$ for the
case of a complete array.

\item
Lemma \logicref{CountEmpty} covers the cases of empty ranges.

\item
Lemmas \logicref{CountHit} and 
\logicref{CountMiss} reduce
counting of a range of length~$n-m$ to a range of length~$n-m-1$.

\item
Lemmas \logicref{CountOne} and \logicref{CountSingle} built on on top of \CountHit
and \CountMiss. 
Using them simplifies several \coq proofs.
They also slightly change the induction scheme from $n-1 \rightarrow n$
to $n \rightarrow n+1$.
\end{itemize}


\begin{logic}[hbt]
\begin{minipage}{\textwidth}
\lstinputlisting[linerange={1-45}, style=acsl-block, frame=single]{Source/Count.acsl}
\end{minipage}
\caption{\label{logic:Count-1}The logic function \Count (1)}
\input{Listings/Count.acsl.labels.tex}
\input{Listings/Count.acsl.index.tex}
\end{logic}

\FloatBarrier

\begin{itemize}

\item
The logic function \Count depends only on the set \inl{a[m..n-1]} of memory locations.
Lemma \logicref{CountUnchanged} makes this claim explicit by ensuring that
\Count produces the same result
if the values \inl{a[0..n-1]} do not change between two program states indicated
by the labels~\inl{K} and~\inl{L}.
We use here predicate \logicref{Unchanged} to express the premise.

\item 
Lemma \logicref{CountEqual} is a generalization of lemma \CountUnchanged for
the case of comparing \Count on two arrays.

\item
Lemmas \logicref{CountUnion} and \logicref{CountCut} 
allow to deal with partitions of arrays.
\end{itemize}

\begin{logic}[hbt]
\begin{minipage}{\textwidth}
\lstinputlisting[linerange={46-90}, style=acsl-block, frame=single]{Source/Count.acsl}
\end{minipage}
\caption{\label{logic:Count-2}The logic function \Count (2)}
\end{logic}

\FloatBarrier

\begin{itemize}
\item 
Lemmas \logicref{CountSingleBounds} and \logicref{CountBounds}
express lower and upper bounds of \Count.
Lemma \logicref{CountIncreasing} states that \Count is a monotonically increasing.

\item
Finally, lemmas \logicref{CountSingleShift} and \logicref{CountShift}
state that \Count is invariant under array shifts.
\end{itemize}

We mention here also lemma \logicref{CountSomeEqual}
which brings together properties of \logicref{Count} and \logicref{Find}.

\input{Listings/CountFind.acsl.tex}

\clearpage 

\subsection{Formal specification of \counti}

In the contract of \specref{counti} we use the logic function
\logicref{Count}
Note that our specification also states that the result of \counti is non-negative
and less than or equal the size of the array.

\input{Listings/count.h.tex}

\subsection{Implementation of \counti}

The following listing shows a possible implementation of \implref{counti}.
Note that we refer to the logic function \Count in one of the loop invariants.

\input{Listings/count.c.tex}

\clearpage

