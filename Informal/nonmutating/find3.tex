
\section{The \findiii algorithm---using a logic function}
\Label{sec:findiii}

In this section we specify linear search yet another way.
This requires more preparing work but results in a more concise function contract.

\subsection{The logic function \Find}

We start with a \emph{recursive} definition of the \acsl function \Find.
Due to the considerable number of associated lemmas of the function
\Find we split its definition into several listings.
Note that \Find comes as two \emph{overloaded} functions.
While the first version is defined for \emph{array sections} the latter is intend
for \emph{complete arrays}.

The listings start with lemmas which express elementary
properties directly related to an incremental increase of the array \inl{a[0..n-1]}. 
The latter lemmas are somewhat more higher-level and will
be useful for the verification of \findiii.
It will be there that we also reuse the predicates 
\logicref{SomeEqual}and
\logicref{NoneEqual}.
%
At the end of this section we will also discuss in what sense the contracts
of \findii and \findiii are equivalent.

\begin{logic}[hbt]
\begin{minipage}{0.99\textwidth}
\lstinputlisting[linerange={1-60}, style=acsl-block, frame=single]{Source/Find.acsl}
\end{minipage}
\caption{\Label{logic:Find-1}The logic function \Find (1)}
\input{Listings/Find.acsl.labels.tex}
\input{Listings/Find.acsl.index.tex}
\end{logic}

\FloatBarrier

\begin{logic}[hbt]
\begin{minipage}{0.99\textwidth}
\lstinputlisting[linerange={61-92}, style=acsl-block, frame=single]{Source/Find.acsl}
\end{minipage}
\caption{\Label{logic:Find-2}The logic function \Find (2)}
\end{logic}

\FloatBarrier

\subsection{Formal specification of \findiii}

Using the logic function \Find we can now give a third specification of linear search.
The contract of \specref{findiii} is considerably shorter than that of \specref{findii}.
Of course, we had to put much more effort into the definition of the \acsl
function \logicref{Find}.

\input{Listings/find3.h.tex}

\clearpage

\subsection{Implementation of \findiii}

The following listing shows the implementation of \implref{findiii}.
In order to achieve a complete verification we had to add the assertion \inl{found}.

\input{Listings/find3.c.tex}

A question that remains is in what sense the contract of \specref{findii} is equivalent to
the one of \specref{findiii}.
We will answer this question in the following section.

\subsection{The equivalence of \findii and \findiii}
\Label{sec:findiv}
\Label{sec:findv}

We consider the contracts of \specref{findii} and \specref{findiii} as \emph{equivalent} 
if each one is sufficient to verify the other.
To this end we introduce yet another two examples \findiv and \findv.

The implementation of \implref{findiv} consists just of a call to \findiii.

\input{Listings/find4.c.tex}

\clearpage

The contract of \specref{findiv}, however, is the same as the one of \specref{findii}.

\input{Listings/find4.h.tex}

Analogously, the implementation of \implref{findv} is simply a call to \findii.

\input{Listings/find5.c.tex}

On the other hand, the contract of \specref{findv} is the same as the one of \specref{findiii}.
%
The verification of the functions \findiv and \findv 
(cf.\ Table~\ref{tbl:result-nonmutating}) then shows the equivalence
of the respective contracts of \specref{findii} and \specref{findiii}.

\input{Listings/find5.h.tex}

\clearpage

