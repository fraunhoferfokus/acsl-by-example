
\chapter{Non-mutating algorithms}
\Label{cha:non-mutating}

\Label{assigns-clause}
In this chapter, we consider \emph{non-mutating} algorithms
of the \cxx Standard Library \cite[\S 28.5]{cxx-17-draft}.
These algorithms neither change their arguments nor any objects outside their scope.
This requirement can be formally expressed with the following 
\emph{assigns clause}:
\begin{lstlisting}[style=acsl-block]

  assigns \nothing;
\end{lstlisting}

Each algorithm in this chapter therefore uses this assigns clause
in its specification.

The specifications of these algorithms are not very complex.
Nevertheless, we have tried to arrange them so that the earlier
examples are simpler than the later ones. 
Each algorithm works on one-dimensional arrays.

\begin{itemize}
\item \find in \S\ref{sec:find}
  provides \emph{sequential} or \emph{linear search}
  and returns the smallest index at which a given value occurs in a given range.
  In \S\ref{sec:findii}, a user-defined \acsl predicate is introduced in order to 
  simplify the reuse of various specification elements.
  We refer to the simplified version as \findii.
  We provide in \S\ref{sec:findiii} a third specification of \find (called \findiii)
  that relies on a user-defined \acsl function that expresses the ideas of 
  linear search on the logic level.

\item \findifnot in \S\ref{sec:findifnot} is a small variation of \find
that searches the first occurrence where a given value does \emph{not} occur.

\item \findfirstof in \S\ref{sec:findfirstof} 
  provides similar to \find a \emph{sequential search}.
  However, unlike \find 
  it does not search for a particular value, 
        but for an arbitrary member of a set.

\item \adjacentfind in \S\ref{sec:adjacentfind}
can be used to find equal neighbors in an array.

\item \equal and \mismatch in \S\ref{sec:equal} are useful for
comparing two ranges element-by-element and identifying where they differ. 

\item \search and \searchn in \S\ref{sec:search} and~\S\ref{sec:searchn}
find a subsequence that is identical to a given sequence
when compared element-by-element and returns the position of the first occurrence.

\item \counti in \S\ref{sec:counti} returns
the number of occurrences of a given value in a range.
Here we will explicitly define a logic function for elements
counting and show that the implementation comply with it.

\item \countii in \S\ref{sec:countii} contains
different specification for the \counti function. In this case an
inductive predicate defined for elements counting. The section allows
one to compare different approaches of writing specifications and
demonstrates the \acsl inductive predicates.

\end{itemize}

\clearpage


\section{The \find algorithm}
\Label{sec:find}

The \find algorithm in the \cxx Standard Library \cite[\S 28.5.5]{cxx-17-draft}
implements \emph{sequential search} for general sequences.
We have modified the generic implementation,
which relies heavily on \cxx templates, to that of a range of
type \valuetype.
The signature now reads:

\begin{lstlisting}[style=acsl-block]

       size_type find(const value_type* a, size_type n, value_type v);
\end{lstlisting}

The function \find returns the least \emph{valid} index \inl{i} of \inl{a}
where the condition \inl{a[i] == v} holds. 
If no such index exists then \find returns the length \inl{n} of the array.


As an example, we consider in Figure~\ref{fig:find} an array.
The arrows indicate which indices will be returned by \find for a given value.
Note that the index~9 points \emph{one past end} of the array.
Values that are not contained in the array are colored in gray.

\begin{figure}[hbt]
\centering
\includegraphics[width=0.60\textwidth]{Figures/find.pdf}
\caption{\Label{fig:find}Some simple examples for \find}
\end{figure}

\FloatBarrier


\subsection{Formal specification of \find}

The following listing shows our first attempt specify \specref{find}.

\input{Listings/find.h.tex}

The \inl{requires}-clause indicates that \inl{n} is non-negative and
that the pointer \inl{a} points to $n$~contiguously allocated objects of type
\valuetype (see~\S\ref{sec:frequentPattern}).
%
The \inl{assigns}-clause indicates that \find (as a non-mutating algorithm),
does not modify any memory location outside its scope (see~Page~\pageref{assigns-clause}).

Generally, we only know that \find returns a non-negative index that is less or
equal the length of the array.
However, once we assume more specific situations,
we can also make more precise statements about the returned valued.
This is the reason why we have subdivided the specification of
\find into two behaviors (named \inl{some} and \inl{none}).

\begin{itemize}
\item
The behavior \inl{some} applies if the sought-after value is contained in the array.
We express this condition by using the \inl{assumes}-clause.
The next line expresses that if the assumptions of the behavior are satisfied then
\find will return a valid index.
The algorithm also ensures that  the returned (valid) index \inl{i},
\inl{a[i] == v} holds.
Therefore we define this property in the second postcondition of behavior \inl{some}.
Finally, it is important to express that \find returns the smallest index~\inl{i}
for which \inl{a[i] == v} holds (see last postcondition of behavior \inl{some}).

\item
The behavior \inl{none} covers the case that the sought-after value 
is \emph{not} contained in the array (see \inl{assumes}-clause of behavior \inl{none} in
in the contract of\specref{find}.
In this case, \find must return the length \inl{n} of the range \inl{a}.
\end{itemize}

Note that the formula in the \inl{assumes}-clause of the behavior \inl{some}
is the negation of the \inl{assumes}-clause of the behavior \inl{none}.
Therefore, we can express  that these two behaviors are
\emph{complete} and \emph{disjoint}.

\subsection{Implementation of \find}

The noteworthy elements of our implementation of \implref{find} are the 
\emph{loop annotations}.
%
The first loop \emph{invariant} is needed to prove that accesses
to~\inl{a} only occur with valid indices. The second  loop \emph{invariant} is needed
for the proof of the postconditions of the
behavior~\inl{some} in the contract of \specref{find}.
It expresses that for each iteration the sought-after value is not
yet found up to that iteration step.
Finally, the  loop \emph{variant} \inl{n-i} is needed to generate correct verification
conditions for the termination of the loop.

\input{Listings/find.c.tex}

\clearpage


\find(2) & \ref{sec:find2} & 19   &  19   & 100  &   9  &   0  &  10  &   0  &   0 \\\hline


\section{The \findiii algorithm---using a logic function}
\Label{sec:findiii}

In this section we specify linear search yet another way.
This requires more preparing work but results in a more concise function contract.

\subsection{The logic function \Find}

We start with a \emph{recursive} definition of the \acsl function \Find.
Due to the considerable number of associated lemmas of the function
\Find we split its definition into several listings.
Note that \Find comes as two \emph{overloaded} functions.
While the first version is defined for \emph{array sections} the latter is intend
for \emph{complete arrays}.

The listings start with lemmas which express elementary
properties directly related to an incremental increase of the array \inl{a[0..n-1]}. 
The latter lemmas are somewhat more higher-level and will
be useful for the verification of \findiii.
It will be there that we also reuse the predicates 
\logicref{SomeEqual}and
\logicref{NoneEqual}.
%
At the end of this section we will also discuss in what sense the contracts
of \findii and \findiii are equivalent.

\begin{logic}[hbt]
\begin{minipage}{0.99\textwidth}
\lstinputlisting[linerange={1-60}, style=acsl-block, frame=single]{Source/Find.acsl}
\end{minipage}
\caption{\Label{logic:Find-1}The logic function \Find (1)}
\input{Listings/Find.acsl.labels.tex}
\input{Listings/Find.acsl.index.tex}
\end{logic}

\FloatBarrier

\begin{logic}[hbt]
\begin{minipage}{0.99\textwidth}
\lstinputlisting[linerange={61-92}, style=acsl-block, frame=single]{Source/Find.acsl}
\end{minipage}
\caption{\Label{logic:Find-2}The logic function \Find (2)}
\end{logic}

\FloatBarrier

\subsection{Formal specification of \findiii}

Using the logic function \Find we can now give a third specification of linear search.
The contract of \specref{findiii} is considerably shorter than that of \specref{findii}.
Of course, we had to put much more effort into the definition of the \acsl
function \logicref{Find}.

\input{Listings/find3.h.tex}

\clearpage

\subsection{Implementation of \findiii}

The following listing shows the implementation of \implref{findiii}.
In order to achieve a complete verification we had to add the assertion \inl{found}.

\input{Listings/find3.c.tex}

A question that remains is in what sense the contract of \specref{findii} is equivalent to
the one of \specref{findiii}.
We will answer this question in the following section.

\subsection{The equivalence of \findii and \findiii}
\Label{sec:findiv}
\Label{sec:findv}

We consider the contracts of \specref{findii} and \specref{findiii} as \emph{equivalent} 
if each one is sufficient to verify the other.
To this end we introduce yet another two examples \findiv and \findv.

The implementation of \implref{findiv} consists just of a call to \findiii.

\input{Listings/find4.c.tex}

\clearpage

The contract of \specref{findiv}, however, is the same as the one of \specref{findii}.

\input{Listings/find4.h.tex}

Analogously, the implementation of \implref{findv} is simply a call to \findii.

\input{Listings/find5.c.tex}

On the other hand, the contract of \specref{findv} is the same as the one of \specref{findiii}.
%
The verification of the functions \findiv and \findv 
(cf.\ Table~\ref{tbl:result-nonmutating}) then shows the equivalence
of the respective contracts of \specref{findii} and \specref{findiii}.

\input{Listings/find5.h.tex}

\clearpage




\section{The \findifnot algorithm }
\label{sec:findifnot}

Many algorithms in the \cxx standard library can be parameterized not only by the type
of sequence but also using so-called \emph{function objects}.
One example is the \findifnot algorithm that accepts a 
\emph{predicate function object}~$P$.
The algorithm then returns the first position~$i$ in the input sequence where
$P(i)$ does \emph{not} hold.

While function objects could be emulated in \isoc with \emph{pointers to functions},
we will not follow this road here.
The main reason is that function pointers are, so far, only supported momentarily by \framac.
Moreover, there is as of now no support for parameterized \acsl predicates.
For these reasons our implementation of \findifnot only returns the first position
in an array where a given value does \emph{not} occur.
The signature, thus, reads 

\begin{lstlisting}[style=acsl-block]

       size_type find_if_not(const value_type* a, size_type n, value_type v);
\end{lstlisting}

On the one hand, this is not a very exciting addition to our
collections of verified algorithms.
It gives us, however, an opportunity to introduce the predicates
\logicref{AllEqual}
and \logicref{SomeNotEqual}
and more importantly the logic function
\logicref{FindNotEqual}
that will later play
an essential role in the specification of the algorithm \removecopy,
or more precisely, its variant \specref{removecopyiii}.

\input{Listings/AllSomeNot.acsl.tex}

\clearpage

The predicate \AllEqual expresses that each member of the array section

\inl{a[m..n-1]} equals~\inl{v}.
We also introduce the predicate \SomeNotEqual which is the negation of \AllEqual.
Both predicates complement the predicates 
\logicref{SomeEqual} and 
\logicref{NoneEqual}.

There are two additional overloaded versions of \AllEqual.
The first version uses the value \inl{a[m]} as \inl{v}.
The second version is just a shortcut when the first index~\inl{m} equals~0.

\subsection{The logic function \FindNotEqual}

The definition of the overloaded logic function \FindNotEqual is shown in
Listings~\ref{logic:FindNotEqual-1} and~\ref{logic:FindNotEqual-2}.
This function is very similar to 
\logicref{Find} except that it
finds the first element in a sequence that \emph{differs} from a given value.
%
Note that in lemma \FindNotEqualUnchanged we are using the predicate \logicref{Unchanged}
that will be defined in a later chapter.

\begin{logic}[hbt]
\begin{minipage}{\textwidth}
\lstinputlisting[linerange={1-41}, style=acsl-block, frame=single]{Source/FindNotEqual.acsl}
\end{minipage}
\caption{\label{logic:FindNotEqual-1}
   The logic function \FindNotEqual (1)}
\input{Listings/FindNotEqual.acsl.labels.tex}
\input{Listings/FindNotEqual.acsl.index.tex}
\end{logic}

\FloatBarrier

\begin{logic}[hbt]
\begin{minipage}{\textwidth}
\lstinputlisting[linerange={42-100}, style=acsl-block, frame=single]{Source/FindNotEqual.acsl}
\end{minipage}
\caption{\Label{logic:FindNotEqual-2}The logic function \FindNotEqual (2)}
\end{logic}

\FloatBarrier

\clearpage

\subsection{Formal specification of \findifnot}

The contract of \specref{findifnot} is, unsurprisingly,
very similar to that of \specref{findiii}.
The only difference is that we replaced 
\logicref{Find} by 
\logicref{FindNotEqual}.

\input{Listings/find_if_not.h.tex}

\subsection{Implementation of \findifnot}

The implementation of \implref{findifnot}
also has a lot of similarities with of \implref{findiii}.
Here again we have replaced \Find by \FindNotEqual and, of course,
we check in the loop body that the value \inl{a[i]} \emph{differs} from the
given value~\inl{v}.

\input{Listings/find_if_not.c.tex}

\clearpage



\section{The \findfirstof algorithm}
\Label{sec:findfirstof}

The \findfirstof algorithm \cite[\S 28.5.7]{cxx-17-draft}
is closely related to \find (see \S\ref{sec:find} and~\S\ref{sec:findii}).


\begin{lstlisting}[style=acsl-block]

  size_type
  find_first_of(const value_type* a, size_type m,
                const value_type* b, size_type n);
\end{lstlisting}

Like \find, it performs a sequential search.
However, while \find searches for a particular value, 
the function
\findfirstof returns the least index \inl{i} such that \inl{a[i]} 
is equal to one of the values \inl{b[0..n-1]}.


\begin{figure}[hbt]
\centering
\includegraphics[width=0.60\textwidth]{Figures/find_first_of.pdf}
\caption{\Label{fig:findfirstof}A simple example for \findfirstof}
\end{figure}

\FloatBarrier


As an example, we consider in Figure~\ref{fig:findfirstof} two arrays.
The arrow indicates the smallest index where one of the elements of the three-element array
occurs.

\subsection{The predicate \HasValueOf}

Similar to our approach in \S\ref{sec:findii}, we define a predicate
\logicref{HasValueOf}
that formalizes the fact that there are valid indices~\inl{i}
and~\inl{j} of the respective arrays~\inl{a} and~\inl{b} such that \inl{a[i] == b[j]} holds.
We have chosen to reuse the predicate
\logicref{SomeEqual} to define \HasValueOf.

\input{Listings/HasValueOf.acsl.tex}

\clearpage

\subsection{Formal specification of \findfirstof}

The following listing shows the formal specification of \findfirstof.
The function contract uses the predicates \logicref{HasValueOf} and
\logicref{SomeEqual} thereby making it
very similar the specification of \specref{findii}.

\input{Listings/find_first_of.h.tex}

\clearpage

\subsection{Implementation of \findfirstof}

Our implementation of \implref{findfirstof} calls \specref{findii},
thereby emphasizing reuse.
Besides, leading to a more concise implementation, we also have to write fewer loop annotations.

\input{Listings/find_first_of.c.tex}

\clearpage


\adjacentfind & \ref{sec:adjacentfind} & 24   &  24   & 100  &  13  &   9  &   0  &   2  &   0 \\\hline

\equal & \ref{sec:equal} & 14   &  14   & 100  &   6  &   7  &   1  &   0  &   0 \\\hline

\search & \ref{sec:search} & 33   &  33   & 100  &  19  &   0  &  14  &   0  &   0 \\\hline

\searchn & \ref{sec:search_n} & 29   &  29   & 100  &  13  &  11  &   0  &   5  &   0 \\\hline


\section{The \findend algorithm}
\Label{sec:findend}

The \findend algorithm in the \cxx Standard Library \cite[\S
28.5.6]{cxx-17-draft} searches for the last
subsequence that is identical to a given sequence when 
compared element-by-element.
For our purposes we have modified
the generic implementation
to that of an array of type \valuetype.
The signature now reads:

\begin{lstlisting}[style = acsl-block]

  size_type
  find_end(const value_type* a, size_type n, const value_type* b, size_type p);
\end{lstlisting}

The function \findend returns the greatest
index \inl{s} of the array \inl{a} where the condition \inl{a[s+k] == b[k]} holds for each
index~\inl{k} with \inl{0 <= k < p}
(see Figure~\ref{fig:findend}).
If no such index exists, then \findend returns the length
\inl{n} of the array \inl{a}. One has to remark the special case \inl{p == 0}.
In this case the last position of the empty string is found (the length \inl{n})
and returned.

\begin{figure}[hbt]
\centering
\includegraphics[width=0.69\textwidth]{Figures/find_end.pdf}
\caption{\Label{fig:findend} Finding the last occurrence \inl{b[0..p-1]} in \inl{a[0..n-1]}}
\end{figure}

\clearpage

\subsection{Formal specification of \findend}

The following listing shows the specification of \specref{findend}.
Conceptually, the specification of the function \findend is very similar to that of
\specref{findii}.
We therefore use again behaviors to capture the essential aspects of \findend.
It is quite clear that these behaviors are \emph{complete} and \emph{disjoint}.

The behavior \inl{has_match} applies if the sequence \inl{a} 
contains a subsequence identical to \inl{b}. 
We express this condition with \inl{assumes} using the predicate 
\logicref{HasSubRange}.
The \inl{ensures} clause \inl{bound} indicates that the return
value must be in the range~\inl{0..n-p}.
The clause \inl{result} of behavior \inl{has_match} expresses that \findend
returns an index where \inl{b} can be found in \inl{a}.
%
Finally, the clause \inl{last}
indicates that the sequence \inl{a} does not contain
\inl{b} beginning at a position larger than \inl{\\result}.

The behavior \inl{no_match} covers the case that there is no
subsequence of \inl{a} that equals
\inl{b}. In this case, \findend must return the length \inl{n} of the
range \inl{a}.

\input{Listings/find_end.h.tex}

\clearpage

\subsection{Implementation of \findend}

Our implementation of \implref{findend} is similar to the one of \implref{search}.

\input{Listings/find_end.c.tex}

We maintain in the variable \inl{r} the prospective value to be
returned, according to the current knowledge.
Initially, it is set to \inl{n}, meaning ``no occurrence of \inl{b}
found yet''.
Whenever an occurrence is found, \inl{r} is updated to its starting
position.

The invariant \inl{bound} states that \inl{r} either still has the value
\inl{n} or has a value up to \inl{n-p}.
For the former case, invariant \inl{not_found}
indicates that no occurrence of \inl{b} has been found.
For the latter case, the loop invariant \inl{found} indicates that an occurrence
\inl{b[0..p-1]} at \inl{r} has indeed been found.
The invariant \inl{last}, on the other hand states that
none was found \emph{after} the index~\inl{r}.

\clearpage


\cnt  & \ref{sec:count} & 19   &  19   & 100  &   8  &   0  &  11  &   0  &   0 \\\hline


\section{The \countii algorithm}
\Label{sec:countii}

In this section, we specify the \counti algorithm in a different way, namely
using the \emph{inductively} defined predicate
\logicref{CountInd} from the
following listing.
%

\input{Listings/CountInd.acsl.tex}

The definition consists of three cases.
\begin{itemize}
\item
The \inl{Nil} case states for arrays of negative pf zero length,
the predicate only holds is \inl{sum} is zero.

\item
The \inl{Hit} and \inl{Miss} define \CountInd for arrays \inl{a[0..n-1]}
of size \inl{n} referring to the array \inl{a[0..n-2]} and the value \inl{a[n-1]}.
\end{itemize}

We remark that the cases are very similar to the
lemmas \logicref{CountEmpty},
\logicref{CountHit}
and \logicref{CountMiss},
except we have use the additional argument \inl{sum} to refer to the number
of counted elements since \CountInd is a predicate.

We  have intentionally used the scheme $n-1 \Rightarrow n$ instead of $n \Rightarrow n+1$.
In this particular case, it allows theorem provers to match loop indices
with premises without additional hints to prove loop invariants.

\subsection{Additional lemmas for the inductive predicate}

The lemmas of 
\logicref{CountIndImplicit}
complement the lemmas of \logicref{Count}.
They demonstrate how existing lemmas can be rewritten for an inductive predicate.
%
These lemmas are not required to prove the \counti function,
but we provide them to complete the illustrative example of how
inductive predicates could be utilized in the specifications.

The inductive definition is the ``complete'' definition
which means that the predicate does not hold for cases outside of its domain of definition.
We state this property explicitly through lemma
\logicref{CountIndInverse}
in the following listing.
Frama-C does not automatically generate this kind of property.
The reason for not adding such a corresponding axiom apparently is that it ``could
confuse first-order theorem provers''.\footnote{\url{https://stackoverflow.com/a/32457870}}

\input{Listings/CountIndImplicit.acsl.tex}

There is also the lemma \logicref{CountIndNonNegative}
which states that the lower bound for the number of the counted elements is zero.
%
The relation between the inductive definition \CountInd and the explicit 
definition of \logicref{Count} is expressed
by lemma \logicref{CountIndCount}.

\input{Listings/CountIndLemmas.acsl.tex}

\clearpage

\subsection{Specification of \countii}

The following listing contains the contracts of \specref{countii}.
It shows the use of the inductive predicate 
\logicref{CountInd}.

\input{Listings/count2.h.tex}

\subsection{Implementation of \countii}

The only difference between the implementation of \implref{countii} 
and the implementation of \implref{counti}
is that we have to supply the value \inl{counted} as an argument
of the predicate \logicref{CountInd}.

\input{Listings/count2.c.tex}

\clearpage



