
\section{The \findfirstof algorithm}
\Label{sec:findfirstof}

The \findfirstof algorithm \cite[\S 28.5.7]{cxx-17-draft}
is closely related to \find (see \S\ref{sec:find} and~\S\ref{sec:findii}).


\begin{lstlisting}[style=acsl-block]

  size_type
  find_first_of(const value_type* a, size_type m,
                const value_type* b, size_type n);
\end{lstlisting}

Like \find, it performs a sequential search.
However, while \find searches for a particular value, 
the function
\findfirstof returns the least index \inl{i} such that \inl{a[i]} 
is equal to one of the values \inl{b[0..n-1]}.


\begin{figure}[hbt]
\centering
\includegraphics[width=0.60\textwidth]{Figures/find_first_of.pdf}
\caption{\Label{fig:findfirstof}A simple example for \findfirstof}
\end{figure}

\FloatBarrier


As an example, we consider in Figure~\ref{fig:findfirstof} two arrays.
The arrow indicates the smallest index where one of the elements of the three-element array
occurs.

\subsection{The predicate \HasValueOf}

Similar to our approach in \S\ref{sec:findii}, we define a predicate
\logicref{HasValueOf}
that formalizes the fact that there are valid indices~\inl{i}
and~\inl{j} of the respective arrays~\inl{a} and~\inl{b} such that \inl{a[i] == b[j]} holds.
We have chosen to reuse the predicate
\logicref{SomeEqual} to define \HasValueOf.

\input{Listings/HasValueOf.acsl.tex}

\clearpage

\subsection{Formal specification of \findfirstof}

The following listing shows the formal specification of \findfirstof.
The function contract uses the predicates \logicref{HasValueOf} and
\logicref{SomeEqual} thereby making it
very similar the specification of \specref{findii}.

\input{Listings/find_first_of.h.tex}

\clearpage

\subsection{Implementation of \findfirstof}

Our implementation of \implref{findfirstof} calls \specref{findii},
thereby emphasizing reuse.
Besides, leading to a more concise implementation, we also have to write fewer loop annotations.

\input{Listings/find_first_of.c.tex}

\clearpage

