
\section{The \countii algorithm}
\Label{sec:countii}

In this section, we specify the \counti algorithm in a different way, namely
using the \emph{inductively} defined predicate
\logicref{CountInd} from the
following listing.
%

\input{Listings/CountInd.acsl.tex}

The definition consists of three cases.
\begin{itemize}
\item
The \inl{Nil} case states for arrays of negative pf zero length,
the predicate only holds is \inl{sum} is zero.

\item
The \inl{Hit} and \inl{Miss} define \CountInd for arrays \inl{a[0..n-1]}
of size \inl{n} referring to the array \inl{a[0..n-2]} and the value \inl{a[n-1]}.
\end{itemize}

We remark that the cases are very similar to the
lemmas \logicref{CountEmpty},
\logicref{CountHit}
and \logicref{CountMiss},
except we have use the additional argument \inl{sum} to refer to the number
of counted elements since \CountInd is a predicate.

We  have intentionally used the scheme $n-1 \Rightarrow n$ instead of $n \Rightarrow n+1$.
In this particular case, it allows theorem provers to match loop indices
with premises without additional hints to prove loop invariants.

\subsection{Additional lemmas for the inductive predicate}

The lemmas of 
\logicref{CountIndImplicit}
complement the lemmas of \logicref{Count}.
They demonstrate how existing lemmas can be rewritten for an inductive predicate.
%
These lemmas are not required to prove the \counti function,
but we provide them to complete the illustrative example of how
inductive predicates could be utilized in the specifications.

The inductive definition is the ``complete'' definition
which means that the predicate does not hold for cases outside of its domain of definition.
We state this property explicitly through lemma
\logicref{CountIndInverse}
in the following listing.
Frama-C does not automatically generate this kind of property.
The reason for not adding such a corresponding axiom apparently is that it ``could
confuse first-order theorem provers''.\footnote{\url{https://stackoverflow.com/a/32457870}}

\input{Listings/CountIndImplicit.acsl.tex}

There is also the lemma \logicref{CountIndNonNegative}
which states that the lower bound for the number of the counted elements is zero.
%
The relation between the inductive definition \CountInd and the explicit 
definition of \logicref{Count} is expressed
by lemma \logicref{CountIndCount}.

\input{Listings/CountIndLemmas.acsl.tex}

\clearpage

\subsection{Specification of \countii}

The following listing contains the contracts of \specref{countii}.
It shows the use of the inductive predicate 
\logicref{CountInd}.

\input{Listings/count2.h.tex}

\subsection{Implementation of \countii}

The only difference between the implementation of \implref{countii} 
and the implementation of \implref{counti}
is that we have to supply the value \inl{counted} as an argument
of the predicate \logicref{CountInd}.

\input{Listings/count2.c.tex}

\clearpage

